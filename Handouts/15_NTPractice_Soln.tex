\documentclass[12pt]{amsart}
\pagestyle{plain}

\usepackage{amsthm, setspace, framed, url, hyperref, lscape}
\usepackage{enumerate}

\usepackage[none]{hyphenat}
\usepackage[pdftex]{graphicx}
\usepackage{enumerate}
\usepackage{tikz}
\usepackage{courier}
\makeatletter
\def\@settitle{\begin{center}%
  \baselineskip14\p@\relax
  \bfseries
  \uppercasenonmath\@title
  \@title
  \ifx\@subtitle\@empty\else
     \\[1ex]\uppercasenonmath\@subtitle
     \footnotesize\mdseries\@subtitle
  \fi
  \end{center}%
}
\def\subtitle#1{\gdef\@subtitle{#1}}
\def\@subtitle{}
\makeatother

\setlength{\textwidth}{6.0in}
\setlength{\textheight}{8.8in}
\setlength{\oddsidemargin}{0.25in}
\setlength{\evensidemargin}{0.25in}
\setlength{\topmargin}{0in}

\theoremstyle{plain}
\newtheorem{thm}{Theorem}
\newtheorem{lem}[thm]{Lemma}
\newtheorem*{cor}{Corollary}
\newtheorem{quest}{Question}
\theoremstyle{definition}
\newtheorem*{defn}{Definition}
\newtheorem*{ex}{Example}
\theoremstyle{remark}
\newtheorem*{rem}{Remark}
\newtheorem*{note}{Note}
\newtheorem{case}{Case}

\begin{document}

\onehalfspacing

\title[]{Cryptography Handout 15}
\subtitle{Number Theory Practice Solutions}
\maketitle

\begin{enumerate}[1.]
	\item Use the Euclidean Algorithm to find the $\gcd$ for the following pairs of numbers:
	\begin{enumerate}[a.]
		\item $\gcd(14129,9353)$
			\begin{framed}
			$\gcd(14129,9353) = 199$
			\end{framed}
		\item $\gcd(30073, 12749)$
			\begin{framed}
			$\gcd(30073, 12749) = 61$
			\end{framed}
	\end{enumerate}
	\item Compute the Euler Phi Function for the following:
	\begin{enumerate}[a.]	
		\item $\varphi(25)$
			\begin{framed}
			20
			\end{framed}
		\item $\varphi(40)$
			\begin{framed}
			16
			\end{framed}	
		\item $\varphi(29)$
			\begin{framed}
			28
			\end{framed}
		\item $\varphi(17)$
			\begin{framed}
			16
			\end{framed}
		\item $\varphi(p)$ where $p$ is a prime
			\begin{framed}
			$p-1$
			\end{framed}
	\end{enumerate}
	\item Use Fermat's Little Theorem to evaluate the following:
		\begin{enumerate}[a.]
			\item $11^{12} \bmod 13$
				\begin{framed}
				$1 \bmod 13$
				\end{framed}
			\newpage \item $11^{13} \bmod 13$
				\begin{framed}
				$11^{12}\cdot 11 \bmod 13 \equiv 1 \cdot 11 \bmod 13 \equiv 11 \bmod 13$
				\end{framed}
			\item $88^{100} \bmod 101$
				\begin{framed}
				$1 \bmod 101$
				\end{framed}
			\item $a^{100} \bmod 101$ for some number $a$
				\begin{framed}
				$1 \bmod 101$
				\end{framed}
			\item $88^{203} \bmod 101$
				\begin{framed}
				${(88^{2}})^{100}\cdot 88^3 \bmod 101 \equiv 88^3 \bmod 101 \equiv 25 \bmod 101$
				\end{framed}
		\end{enumerate}
		\item Use Euler's Theorem to evaluate the following:
		\begin{enumerate}[a.]
			\item $23^{20} \bmod 25$
				\begin{framed}
				$1 \bmod 25$
				\end{framed}
			\item $23^{21} \bmod 25$
				\begin{framed}
				$23^{20}\cdot 23 \bmod 25 \equiv 1 \cdot 23 \bmod 25 \equiv 23 \bmod 25$
				\end{framed}
			\item $31^{16} \bmod 40$
				\begin{framed}
				$1 \bmod 40$
				\end{framed}
			\item $a^{16} \bmod 40$ for some number $a$
				\begin{framed}
				$1 \bmod 40$
				\end{framed}
			\item $17^{55} \bmod 40$
				\begin{framed}
				${(17^{3}})^{16}\cdot 17^7 \bmod 40 \equiv 17^7 \bmod 40 \equiv 33 \bmod 40$
				\end{framed}
		\end{enumerate}
		\item Determine the order of the following numbers $a$ and primes $p$ (recall the order is the smallest power $k$ in which $a^k \equiv 1 \bmod p$):
		\begin{enumerate}[a.]
			\item $a = 3, p = 7$
				\begin{framed}
				$k = 6$
				\end{framed}
			\newpage \item $a = 2, p = 7$
				\begin{framed}
				$k = 3$
				\end{framed}
			\item $a = 3, p = 23$
				\begin{framed}
				$k = 11$
				\end{framed}
			\item $a = 7, p = 13$
				\begin{framed}
				$k = 12$
				\end{framed}
		\end{enumerate}         
		\item In the previous question, which values are primitive roots (i.e. the order is $p-1$)?
			\begin{framed}
			$3$ is a primitive root when $p =  7$.  Also, $7$ is a primitive root when $p = 13$.
			\end{framed}
		
		\item Given an integer $a$ and an odd prime $p$, determine if $a$ is a square mod $p$ (use Euler's Criterion).
		\begin{enumerate}[a.]
			\item $a = 3, p = 7$
				\begin{framed}
				$3^{\frac{7-1}{2}} \equiv -1 \bmod 7$ so 3 is not a square mod 7.
				\end{framed}
			\item $a = 10, p = 13$
				\begin{framed}
				$10^{\frac{13-1}{2}} \equiv 1 \bmod 13$ so 10 is  a square mod 13.
				\end{framed}
			\item $a = 10, p = 17$
				\begin{framed}
				$10^{\frac{17-1}{2}} \equiv -1 \bmod 7$ so 10 is not a square mod 17.
				\end{framed}
			\item $a = 45, p = 199$
				\begin{framed}
				$45^{\frac{199-1}{2}} \equiv 1 \bmod 199$ so 45 is  a square mod 199.
				\end{framed}
		\end{enumerate}
		\item Use the Legendre symbol $\left( \frac{a}{p}\right)$ to determine whether $a = -1$ is a square or not for the following primes $p$:
		\begin{enumerate}[a.]
			\item $p = 17$
				\begin{framed}
				$17 \bmod 4 \equiv 1 \bmod 4$, so it is a square.
				\end{framed}
			\newpage \item $p =  59$
				\begin{framed}
				$59 \bmod 4 \equiv 3 \bmod 4$, so it is not a square.
				\end{framed}
			\item $p = 83$
				\begin{framed}
				$83 \bmod 4 \equiv 3 \bmod 4$, so it is not a square.
				\end{framed}
		\end{enumerate}
		\item First, complete the following table.  Then use Euler's Criterion and Quadratic Reciprocity to determine the next questions.
			\begin{center}
			\begin{tabular}{|p{1in}|p{3in}|} \hline
			Prime $p$ & Congruent to $1 \bmod 4$ or $3 \bmod 4$?\\ \hline
			19 & 3\\
			29 & 1\\
			61 & 1\\
			67 & 3\\ \hline
			\end{tabular}
			\end{center}
		\begin{enumerate}[a.]
			\item $\left(\frac{19}{29}\right)$
				\begin{framed}
				-1
				\end{framed}
			\item $\left(\frac{29}{19}\right)$
				\begin{framed}
				-1
				\end{framed}
			\item $\left(\frac{29}{61}\right)$
				\begin{framed}
				-1
				\end{framed}
			\item $\left(\frac{61}{29}\right)$
				\begin{framed}
				-1
				\end{framed}
			\item $\left(\frac{67}{19}\right)$
				\begin{framed}
				-1
				\end{framed}
			\item $\left(\frac{19}{67}\right)$
				\begin{framed}
				1
				\end{framed}
		\end{enumerate}
\end{enumerate}

\end{document}