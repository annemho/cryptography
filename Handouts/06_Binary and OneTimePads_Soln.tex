\documentclass[12pt]{amsart}
\pagestyle{plain}

\usepackage{amsthm, setspace, framed, url, hyperref, lscape}
\usepackage[none]{hyphenat}
\usepackage[pdftex]{graphicx}
\usepackage{enumerate}
\usepackage{tikz}
\usepackage{courier}
\makeatletter
\def\@settitle{\begin{center}%
  \baselineskip14\p@\relax
  \bfseries
  \uppercasenonmath\@title
  \@title
  \ifx\@subtitle\@empty\else
     \\[1ex]\uppercasenonmath\@subtitle
     \footnotesize\mdseries\@subtitle
  \fi
  \end{center}%
}
\def\subtitle#1{\gdef\@subtitle{#1}}
\def\@subtitle{}
\makeatother

\setlength{\textwidth}{6.0in}
\setlength{\textheight}{8.8in}
\setlength{\oddsidemargin}{0.25in}
\setlength{\evensidemargin}{0.25in}
\setlength{\topmargin}{0in}

\theoremstyle{plain}
\newtheorem{thm}{Theorem}[section]
\newtheorem{lem}[thm]{Lemma}
\newtheorem*{cor}{Corollary}
\newtheorem{quest}{Question}

\theoremstyle{definition}
\newtheorem*{defn}{Definition}
\newtheorem*{ex}{Example}

\theoremstyle{remark}
\newtheorem*{rem}{Remark}
\newtheorem*{note}{Note}
\newtheorem{case}{Case}

\newcommand{\R}{\mathbb{R}}
\newcommand{\Z}{\mathbb{Z}}
\newcommand{\C}{\mathbb{C}}
\newcommand{\N}{\mathbb{N}}
\newcommand{\QQ}{\mathbb{Q}}
\newcommand{\Rnn}{\R^{n\times n}} 
\newcommand{\Rn}{\R^{n}} 

\newcommand{\bx}{{\bf x}}
\newcommand{\bv}{{\bf v}}
\newcommand{\bw}{{\bf w}}
\newcommand{\bu}{{\bf u}}

\newcommand{\bit}{\begin{itemize}}
\newcommand{\eit}{\end{itemize}}
\newcommand{\ben}{\begin{enumerate}}
\newcommand{\een}{\end{enumerate}}
\newcommand{\bea}{\begin{eqnarray*}}
\newcommand{\eea}{\end{eqnarray*}}
\newcommand{\bpf}{\begin{proof}}
\newcommand{\epf}{\end{proof}\ms}
\newcommand{\bthm}{\begin{thm}}
\newcommand{\ethm}{\end{thm}}
\newcommand{\bdefn}{\begin{defn}}
\newcommand{\edefn}{\end{defn}}
\newcommand{\bex}{\begin{ex}}
\newcommand{\eex}{\end{ex}}
\newcommand{\bde}{\begin{description}}
\newcommand{\ede}{\end{description}}
\newcommand{\bcen}{\begin{center}}
\newcommand{\ecen}{\end{center}}
\newcommand{\bq}{\begin{quest}}
\newcommand{\eq}{\end{quest}}

\newcommand*\circled[1]{\tikz[baseline=(char.base)]{
            \node[shape=circle,draw,inner sep=2pt] (char) {#1};}}

\begin{document}

\onehalfspacing

\title[]{Cryptography Handout 06}
\subtitle{Binary and One-Time Pads Solutions}
\maketitle

\section{Binary} Convert the following numbers to binary:
\begin{enumerate}[1.]
	\item 58 =111010\\
	\item 47 = 101111\\ 
	\item 30 = 11110\\  
	\item 31 = 11111\\  
\end{enumerate}

\noindent \emph{Question: What do you notice about the right-most bit for even versus odd numbers?}\\ 
Even numbers have right-most bit being a 0 whereas odd numbers have right-most bit being a 1.


\section{Pseudo-Random Bit Generation}
\subsection*{Linear Congruential Generator}
\begin{enumerate}[1.]
	\item Start with an initial seed $x_0$.
	\item Generate a sequence of numbers $x_1, x_2, \cdots$ in which $x_n = a x_{n-1}+b \bmod m$ for parameters $a,b$ and $m$.
\end{enumerate}

\bex
Suppose $x_0 = 2, a = 3, b = 7, m = 5$.  Generate $x_1, x_2, x_3, x_4, x_5$.\\ 
\begin{align*}
x_1 &= 3\\
x_2 &= 1\\
x_3 &= 0\\
x_4 &= 2\\
x_5 &= 3\\
\end{align*}

\noindent \emph{Question: What are potential problems with this method?}\\ 
We chose small digits in this example, and we know $a,b,m$, so the numbers are obviously predictable. Even if $a, b, m$ aren't known, the numbers are still fairly predictable with a high probability.\\
\eex


\subsection*{Blum-Blum-Shub (BBS) Pseudo-Random Bit Generator}
\begin{enumerate}[1.]
	\item Choose two large prime numbers $p$ and $q$ in which both primes are congruent to $3 \bmod 4$.  Multiply them together to get $n = pq$.
	\item Choose a seed $x_0 \equiv x^2 \bmod n$ where $x$ is a number in which $\gcd(n,x) = 1$.
	\item Generate a sequence of numbers $b_1, b_2, \cdots$ in which
		\begin{enumerate}[a.]
			\item $x_j \equiv x_{j-1}^2 \bmod n$
			\item $b_j$ is the least significant bit or right-most bit of $x_j$
		\end{enumerate}
\end{enumerate}

\bex
(from T \& W 2.10) Suppose we choose $p = 24672462467892469787$ and $q = 396736894567834589803$.  Then $$n = 9788476140853110794168855217413715781961.$$

One possible choice of $x$ is $x = 873245647888478349013$.

The initial seed is $x_0 = 8845298710478780097089917746010122863172$.\\

We can compute $x_1, x_2, \cdots$ to get:
\begin{align*}
x_1 &\equiv 7118894281131329522745962455498123822408\\
x_2 & \equiv 3145174608888893164151380152060704518227\\
x_3 & \equiv 4898007782307156233272233185574899430355\\
x_4 & \equiv 3935457818935112922347093546189672310389\\
x_5 & \equiv 675099511510097048901761303198740246040\\
\end{align*}

What are $b_1, b_2, b_3, b_4, b_5$?\\
\begin{align*}
b_1 &= 0\\
b_2 &= 1\\
b_3 &= 1\\
b_4 &= 1\\
b_5 &= 0\\
\end{align*}
\eex

\section{One-Time Pads}
\begin{enumerate}[1.]
	\item Write your plaintext as a sequence of 0s and 1s (binary, or use ASCII, etc.).
	\item The key is a random sequence of 0s and 1s of the same length as the message.  Once a key is used, it is discarded and not used again.
	\item Encryption: Add the key to the message using XOR (exclusive or) or adding $\bmod 2$.
	\item Decryption: Add the same key to the ciphertext to get the plaintext back.
\end{enumerate}

\bex
\texttt{10110011+11111111=01001100}
\eex

\subsection*{Variation}
\begin{enumerate}[1.]
	\item Write your plaintext message.
	\item Count the number $n$ of letters you have in your plaintext.
	\item The key is a random sequence of shifts between 0 and 25.  On SageMathCell, type \texttt{randint(0,25)}.  Make sure your key is as long as the message (so there are $n$ shifts).
	\item Encryption: follow your sequence of shifts.
	\item Decryption: subtract instead of adding your shifts.
\end{enumerate}

\noindent \emph{Question: How might you go about sending your key to a recipient of your message securely?}\\ 
\vspace{1in}

\newpage Now your goal today is to work individually on the following:
\begin{enumerate}[1.]
	\item Create a plaintext message (about 10 letters long) of length $n$.
	\item Generate a sequence of $n$ shifts between 0 and 25.  This is your key.
	\item Use your sequence of shifts to encrypt your message.
	\item Find a way to deliver your encrypted message to your intended recipient.  They will do the same and will try to give you a message.  Remember that your recipient must get a copy of your key as well as your ciphertext!  You are not allowed to just hand them your plaintext.
	\item Decrypt the message that you are given.
\end{enumerate}

\begin{center}
Alternatively...
\end{center}

\noindent If you don't want to go through all that work, you can intercept someone else's message, and if you decrypt it before the intended recipient, you win.

\subsection*{Discussion}
\begin{itemize}
	\item Which issues did you come across when you were trying to deliver your message?\\
	\item How might this generalize to issues in the `real world?'\\
	\item Do you have any ideas on modifying the one-time pad to make the delivery of the key more secure?\\
\end{itemize}


\end{document}