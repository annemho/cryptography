\documentclass[12pt]{amsart}
\pagestyle{plain}

\usepackage{amsthm, setspace, framed, url, hyperref, lscape}
\usepackage{enumerate}
\usepackage{listings}


\usepackage[none]{hyphenat}
\usepackage[pdftex]{graphicx}
\usepackage{tikz}
\usepackage{courier}
\makeatletter
\def\@settitle{\begin{center}%
  \baselineskip14\p@\relax
  \bfseries
  \uppercasenonmath\@title
  \@title
  \ifx\@subtitle\@empty\else
     \\[1ex]\uppercasenonmath\@subtitle
     \footnotesize\mdseries\@subtitle
  \fi
  \end{center}%
}
\def\subtitle#1{\gdef\@subtitle{#1}}
\def\@subtitle{}
\makeatother

\setlength{\textwidth}{6.0in}
\setlength{\textheight}{8.8in}
\setlength{\oddsidemargin}{0.25in}
\setlength{\evensidemargin}{0.25in}
\setlength{\topmargin}{0in}

\theoremstyle{plain}
\newtheorem{thm}{Theorem}
\newtheorem{lem}[thm]{Lemma}
\newtheorem*{cor}{Corollary}
\newtheorem{quest}{Question}
\theoremstyle{definition}
\newtheorem*{defn}{Definition}
\newtheorem*{ex}{Example}
\theoremstyle{remark}
\newtheorem*{rem}{Remark}
\newtheorem*{note}{Note}
\newtheorem{case}{Case}

\begin{document}

\onehalfspacing

\title[]{Cryptography Handout 18}
\subtitle{Digital Signatures (Guided Notes)}
\maketitle

\section{RSA Signatures}
\begin{framed}
Bob has a document or message $m$ that Alice agrees to sign.
\begin{enumerate}[1.]
	\item Signing process
		\begin{enumerate}[a.]
			\item Alice generates two large primes $p$ and $q$.  She computes $n = pq$.
			\item She chooses $e_A$ where $1 < e_A < \varphi(n)$ with $\gcd(e_A, \varphi(n)) = 1$.
			\item She computes $d_A$ such that $e_Ad_A \equiv 1 \bmod \varphi(n)$.
			\item Alice publishes $(e_A, n)$ and keeps $d_A, p, q$ private.
			\item Her signature is $y \equiv m^{d_A} \bmod n$.  $(m, y)$ are made public.
		\end{enumerate}
	\item Verification process
	\begin{enumerate}[a.]
		\item Bob gets Alice's $(e_A, n)$.  He computes $z \equiv y^{e_A} \bmod n$.
		\item If $z = m$, then Bob accepts the signature as valid.  Otherwise, the signature is not valid.
	\end{enumerate}
\end{enumerate}
\end{framed}

\begin{ex}
\begin{enumerate}[1.]
	\item[] $m = 35$
	\item Signing process: $p = 7, q = 13$
		\begin{enumerate}[a.]
			\item $n = \underline{\hspace{1in}}$
			\item $\varphi(n) = \underline{\hspace{1in}}$
				$e_A = \underline{\hspace{1in}}$
			\item $d_A = \underline{\hspace{1in}}$
			\item Public info: $(e_A, n)= \underline{\hspace{1in}}$
			\item Alice's Signature: $y = \underline{\hspace{1in}}$
		\end{enumerate}
	\item Verification Process: Bob sees $(e_A, n)$ and $(m,y) = \underline{\hspace{1in}}$.
		\begin{enumerate}
			\item He computes $z = \underline{\hspace{1in}}$
			\item Is the signature valid or not?
		\end{enumerate}
\end{enumerate}
\vspace{1in}
\end{ex}

\newpage \begin{ex}
\begin{enumerate}[1.]
	\item[] $m = 14$
	\item Signing process: $p = 11, q = 17$
		\begin{enumerate}[a.]
			\item $n = \underline{\hspace{1in}}$
			\item $\varphi(n) = \underline{\hspace{1in}}$
				$e_A = 7$ works here because $\gcd(7,\varphi(n)) =  \underline{\hspace{1in}}$
			\item $d_A =183$ works here because $e_Ad_A \equiv 1 \bmod \varphi(n)$.  Check this:  $\underline{\hspace{1in}}$
			\item Public info: $(e_A, n)= \underline{\hspace{1in}}$
			\item Alice's Signature: $y = \underline{\hspace{1in}}$
		\end{enumerate}
	\item Verification Process: Bob sees $(e_A, n)$ and $(m,y) = \underline{\hspace{1in}}$.
		\begin{enumerate}
			\item He computes $z = \underline{\hspace{1in}}$
			\item Is the signature valid or not?\\
		\end{enumerate}
\end{enumerate}
What if during the verification process, Bob had received $(m,y) = (14,158)$ instead?  What would he conclude?
\vspace{.5in}
\end{ex}

\section{Blind Signatures - RSA}
In some cases, a message is ``blinded" or disguised before it is signed.


\begin{framed}
\begin{enumerate}[1.]
	\item Alice chooses two primes $p$ and $q$.  Then she computes $n = pq$.
	\item Alice also chooses an encryption exponent $e$ and decryption exponent $d$.
	\item $(n,e)$ are public whereas $p,q,d$ are private.
	\item Bob chooses a random integer $k \bmod n$ with $\gcd(k,n) = 1$ and computes $t \equiv k^e m \bmod n$.  He sends $t$ to Alice.
	\item Alice signs $t$ by computing $s \equiv t^d \bmod n$.  She gives $s$ to Bob.
	\item Bob computes $s/k \bmod n$, which is $m^d$.
\end{enumerate}
\end{framed}

\begin{ex}
\begin{enumerate}[1.]
	\item[] $m = 11$
	\item $p = 7, q = 13$, so $n = pq = \underline{\hspace{1in}}$
	\item $e = 5$ and $d = 29$ because $de \equiv 1 \bmod \varphi(n)$.  Verify this:  $\underline{\hspace{1in}}$
	\item $(n,e) = \underline{\hspace{1in}}$
	\item $k = \underline{\hspace{1in}}$ since $\gcd(k,n) = 1$.  He computes $t = \underline{\hspace{1in}}$
	\item $s = \underline{\hspace{1in}}$
	\item $s/k = \underline{\hspace{1in}}$ which should match up with $m^d = \underline{\hspace{1in}}$
\end{enumerate}
\end{ex}

\newpage \begin{ex}
\begin{enumerate}[1.]
	\item[] $m = 23$
	\item $p = 11, q = 17$, so $n = pq = \underline{\hspace{1in}}$
	\item $e = 7$ and $d = 183$ because $de \equiv 1 \bmod \varphi(n)$.  Verify this:  $\underline{\hspace{1in}}$
	\item $(n,e) = \underline{\hspace{1in}}$
	\item $k = \underline{\hspace{1in}}$ since $\gcd(k,n) = 1$.  He computes $t = \underline{\hspace{1in}}$
	\item $s = \underline{\hspace{1in}}$
	\item $s/k = \underline{\hspace{1in}}$ which should match up with $m^d = \underline{\hspace{1in}}$
\end{enumerate}
\end{ex}

\begin{quest}
Show that $s/k$ is actually the signed message $m^d$.\\ \vspace{2in}
\end{quest}

\section{ElGamal Signature Scheme}
The ElGamal Encryption method can also be modified to give a signature scheme.

\begin{framed}
Before she gets started, Alice chooses a prime $p$ and a primitive root $\alpha$.  She chooses a secret integer $a$ such that $1 \leq a \leq p-2$ and calculates $\beta \equiv \alpha^a \bmod p$.  $(p, \alpha, \beta)$ are made public while $a$ is private.
\begin{enumerate}[1.]
	\item Signing process
		\begin{enumerate}[a.]
			\item Alice chooses a secret random $k$ such that $\gcd(k,p-1) = 1$.
			\item She computes $r \equiv \alpha^k \bmod p \text{ with } 0 < r < p$.
			\item She also computes $s \equiv k^{-1}(m-ar) \bmod (p-1).$  The signed message is $(m,r,s)$.
		\end{enumerate}
	\item Verification process
	\begin{enumerate}[a.]
		\item Bob gets Alice's public key $(p, \alpha, \beta)$.
		\item He computes $v_1 \equiv \beta^r r^s \bmod p$ and $v_2 \equiv \alpha^m \bmod p$.
		\item The signature is valid if and only if $v_1 \equiv v_2 \bmod p$.
	\end{enumerate}
\end{enumerate}
\end{framed}

\newpage \begin{ex}
Before she gets started, Alice chooses a prime $p = 17$ and a primitive root $\alpha = 3$.  She chooses a secret integer $a = 4$ such that $1 \leq a \leq p-2$ and calculates $\beta \equiv \alpha^a \bmod p = \underline{\hspace{1in}}$.  $(p, \alpha, \beta) = \underline{\hspace{1in}}$ are made public while $a$ is private.
\begin{enumerate}[1.]
	\item Signing process
		\begin{enumerate}[a.]
			\item $k = 5$ since $\gcd(k,p-1) = 1$.  Verify this: $\underline{\hspace{1in}}$.
			\item $r = \underline{\hspace{1in}}$
			\item $s = \underline{\hspace{1in}}$.  The signed message is $(m,r,s) = \underline{\hspace{1in}}$.
		\end{enumerate}
	\item Verification process
	\begin{enumerate}[a.]
		\item Bob gets Alice's public key $(p, \alpha, \beta)$.
		\item $v_1 = \underline{\hspace{1in}}$ and $v_2 = \underline{\hspace{1in}}$.
		\item The signature is valid if and only if $v_1 \equiv v_2 \bmod p$.
	\end{enumerate}
\end{enumerate}
\end{ex}

\begin{quest}
Show that the verification process works.  Assume the signature is valid with the following steps:\\
\begin{itemize}
	\item Since $s \equiv k^{-1}(m-ar) \bmod p-1$, then $sk \equiv \underline{\hspace{1in}} \bmod (p-1)$.\\
	\item This means $m \equiv  \underline{\hspace{1in}} \bmod (p-1)$.\\
	\item A congruence $\mod p-1$ in the exponent yields an overall congruence $\mod p$, so we have:
\end{itemize}
\end{quest}

\end{document}