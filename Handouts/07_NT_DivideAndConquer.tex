\documentclass[12pt]{amsart}
\pagestyle{plain}

\usepackage{amsthm, setspace, framed, url, hyperref, lscape}
\usepackage[none]{hyphenat}
\usepackage[pdftex]{graphicx}
\usepackage{enumerate}
\usepackage{tikz}
\usepackage{courier}
\makeatletter
\def\@settitle{\begin{center}%
  \baselineskip14\p@\relax
  \bfseries
  \uppercasenonmath\@title
  \@title
  \ifx\@subtitle\@empty\else
     \\[1ex]\uppercasenonmath\@subtitle
     \footnotesize\mdseries\@subtitle
  \fi
  \end{center}%
}
\def\subtitle#1{\gdef\@subtitle{#1}}
\def\@subtitle{}
\makeatother

\setlength{\textwidth}{6.0in}
\setlength{\textheight}{8.8in}
\setlength{\oddsidemargin}{0.25in}
\setlength{\evensidemargin}{0.25in}
\setlength{\topmargin}{0in}

\theoremstyle{plain}
\newtheorem{thm}{Theorem}[section]
\newtheorem{lem}[thm]{Lemma}
\newtheorem*{cor}{Corollary}
\newtheorem{quest}{Question}
\theoremstyle{definition}
\newtheorem*{defn}{Definition}
\newtheorem*{ex}{Example}

\begin{document}

\onehalfspacing

\title[]{Cryptography Handout 07}
\subtitle{``Divide and Conquer"}

\maketitle

\begin{center}
\emph{(adapted from \underline{Number Theory Through Inquiry} by Marshall, Odell, and Starbird)}
\end{center}

\section{Divisibility}
In addition to understanding mathematical concepts, we need to practice communicating our understanding.  The next exercises will help you to practice writing mathematical explanations.\\

\begin{ex}
Here is an example of a mathematical statement (which we write as a \emph{theorem}) with a formal explanation of why it is true (written as a \emph{proof}).
\begin{thm}
Let $n$ be an integer.  If $6 \mid n$, then $3 \mid n$.
\end{thm}
\begin{proof}
Suppose that $6 \mid n$.  Then by definition, there exists an integer $k$ such that $n = 6k$.  We want to show that there exists another integer $k'$ such that $n = 3k'$.  Since $n = 6k = 3(2k)$, then we can choose $k' = 2k$, which concludes that $3 \mid n$.
\end{proof}
\end{ex}

\begin{center}\textbf{Now it's your turn to try writing some proofs.  Instructions:}\end{center}
\begin{enumerate}[1.]
	\item Read the theorem.
	\item Come up with two examples with your partner(s) so that you believe it is true.
	\item Individually write a mathematical proof (most of these will draw on definitions).
	\item Switch papers and read your partner's proof.  Discuss the good parts as well as anything that is confusing.  \emph{Don't take any criticism of your writing personally!  Everyone is here to help each other, and the feedback is meant to be constructive.}
\end{enumerate}

\begin{thm}
Let $a, b, c$ be integers.  If $a \mid b$ and $a \mid c$, then $a \mid (b+c).$
\end{thm}

\begin{thm}
Let $a, b, c$ be integers.  If $a \mid b$ and $a \mid c$, then $a \mid bc$.
\end{thm}

\begin{thm}
Let $a, b, c, n$ be integers with $n > 0$.  If $a \equiv b \bmod n$ and $b \equiv c \bmod n$, then $a \equiv c \bmod n$.
\end{thm}

\newpage \section{Euclidean Algorithm}

First, let's get more familiar with the Euclidean Algorithm by doing some examples.  Run through the algorithm, and write down the steps for the following pairs of numbers.

\begin{enumerate}[1.]
	\item $a = 96, b = 112$
	\item $a = 162, b = 31$
\end{enumerate}

Recall that two integers $a$ and $b$ are \emph{relatively prime} if $\gcd(a,b) = 1$.  It can be shown that an equivalent statement is that we can write that there exists integers $x$ and $y$ such that $ax+by = 1$.\\

Let's do some more proof-writing practice.

\begin{thm}
Let $a, b, c$ be integers.  If $a \mid bc$ and $\gcd(a,b) = 1$, then $a \mid c$.
\end{thm}

\begin{thm}
Let $a, b, n$ be integers.  If $\gcd(a,n) = 1$ and $\gcd(b,n) = 1$, then $\gcd(ab,n) = 1$.
\end{thm}

\end{document}