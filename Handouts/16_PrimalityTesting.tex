\documentclass[12pt]{amsart}
\pagestyle{plain}

\usepackage{amsthm, setspace, framed, url, hyperref, lscape}
\usepackage{enumerate}
\usepackage{listings}


\usepackage[none]{hyphenat}
\usepackage[pdftex]{graphicx}
\usepackage{tikz}
\usepackage{courier}
\makeatletter
\def\@settitle{\begin{center}%
  \baselineskip14\p@\relax
  \bfseries
  \uppercasenonmath\@title
  \@title
  \ifx\@subtitle\@empty\else
     \\[1ex]\uppercasenonmath\@subtitle
     \footnotesize\mdseries\@subtitle
  \fi
  \end{center}%
}
\def\subtitle#1{\gdef\@subtitle{#1}}
\def\@subtitle{}
\makeatother

\setlength{\textwidth}{6.0in}
\setlength{\textheight}{8.8in}
\setlength{\oddsidemargin}{0.25in}
\setlength{\evensidemargin}{0.25in}
\setlength{\topmargin}{0in}

\theoremstyle{plain}
\newtheorem{thm}{Theorem}
\newtheorem{lem}[thm]{Lemma}
\newtheorem*{cor}{Corollary}
\newtheorem{quest}{Question}
\theoremstyle{definition}
\newtheorem*{defn}{Definition}
\newtheorem*{ex}{Example}
\theoremstyle{remark}
\newtheorem*{rem}{Remark}
\newtheorem*{note}{Note}
\newtheorem{case}{Case}

\begin{document}

\onehalfspacing

\title[]{Cryptography Handout 16}
\subtitle{Primality Testing}
\maketitle

\section{First Explorations}
Recall that a \textbf{prime number} $p$ is one in which the only factors are 1 and $p$.

\begin{ex}
$p = 7, 31, 73, 311$ are prime numbers.
\end{ex}

\begin{quest}
Consider the following numbers: $292, 293, 299, 313, 427, 757, 759, 290829273$.\\
Determine if they are prime or not.\\
Discuss your strategies and any problems that arise with your group.
\end{quest}


\newpage \section{Factoring is not the same as Primality Testing}
\noindent The following theorem is called the ``basic principle" in your text (6.3 p. 176).
\begin{thm}
Let $n$ be an integer and suppose there exist integers $x$ and $y$ with $$x^2 \equiv y^2 \bmod n,$$ but $x \not\equiv \pm y \bmod n$.  Then $n$ is composite.  Moreover, $\gcd(x-y,n)$ gives a nontrivial factor of $n$.
\end{thm}

\begin{ex}
Let $x = 12$, $y = 2$ and $n = 35$ in the theorem.  Work through the example and convince yourself the theorem works.  Be sure to find a nontrivial factor of $n$ too.\\ \vspace{2in}
\end{ex}

The next part will walk through the proof of the ``basic principle."  Work as a group to understand every line of the proof.  First, let's prove a lemma:

\begin{lem}
Given integers $a, b$ and $c$.  Suppose $a \mid bc$ and $\gcd(a,b) = 1$.  Then $a \mid c$. 
\end{lem}
\textbf{[Prove this lemma before you move on.]}\\ \vspace{2.5in}


\newpage Now we'll prove the theorem.
\begin{proof}
Let $d = \gcd(x-y,n)$.\\


Case one: If $d = n$, then $x \equiv y \bmod n$.  \textbf{[Convince yourself this is true before you move on.]}\\  \vspace{1in}


This is assumed not to happen, so $d \neq n$.\\

Case two: Suppose $d = 1$.  Using the lemma, we can show that since $n$ divides $x^2-y^2 = (x-y)(x+y)$, then we must have that $n$ divides $x+y$.  \textbf{[Convince yourself this is true before you move on.]}\\
\vspace{2in}

This is a contradiction since we assumed $x \not\equiv -y \bmod n$.  Therefore, $d \neq 1$ and $d \neq n$, so $d$ is a nontrivial factor of $n$.\\ \vspace{.5in}
\end{proof}


\begin{note}
It turns out that showing a number is composite is an easier problem than it is to factor it, so the two problems are not the same.
\end{note}

\newpage \begin{ex}
Recall Fermat's Little Theorem.  It shows that if $p$ is prime, then $2^{p-1} \equiv 1 \bmod p$ \textbf{[Why is this true?]}.  We can use this to show that 35 is not prime without finding a factor by using \emph{successive squaring}:
\begin{align*}
2^4 &\equiv 16\\
2^8 &\equiv 256 \equiv 11\\
2^{16} &\equiv 121 \equiv 16\\
2^{32} &\equiv 256 \equiv 11\\
\end{align*}
Thus, we know that $2^{34} \equiv 2^{32}2^{2} \equiv 11 \cdot 4 \equiv 8 \not\equiv 1 \mod 35.$  Using the above result from Fermat's Little Theorem, we see that $35$ cannot be prime, so it must be composite.
\end{ex}

\section{Primality Tests (which should be called Compositeness Tests)} The following tests give us different ways to test if a number is prime or not.
\subsection{Fermat Primality Test} Let $n > 1$ be an integer.  Choose a random integer $a$ with $1 < a < n-1$.  If $a^{n-1} \not\equiv 1 \bmod n$, then $n$ is composite.  If $a^{n-1} \equiv 1 \bmod n$, then $n$ is probably prime.\\

\noindent Use the Fermat Primality Test on the following small examples to verify that they probably are prime or are composite.
\begin{enumerate}[1.]
	\item $n = 292$\\ \vspace{.2in}
	\item $n = 299$\\ \vspace{.2in}
	\item $n = 757$\\ \vspace{.2in}
\end{enumerate}
Some short SageMath code to do this would be:
\begin{lstlisting}
	n = 757
	a = 3
	a^(n-1)%n
\end{lstlisting}

\subsection{Solovay-Strassen Primality Test} Let $n$ be an odd integer.  Choose several random integers $a$ with $1 < a < n-1$.  If the Jacobi symbol is such that $$\left( \frac{a}{n} \right) \not\equiv a^{\frac{n-1}{2}} \bmod n$$
for some $a$, then $n$ is composite.  If 
$$\left( \frac{a}{n} \right)  \equiv a^{\frac{n-1}{2}} \bmod n$$
for all $a$, then $n$ is probably prime.\\

\noindent Use the Solovay-Strassen Primality Test on the following small examples to verify that they probably are prime or are composite.
\begin{enumerate}[1.]
	\item $n = 293$\\ \vspace{.2in}
	\item $n = 313$\\ \vspace{.2in}
	\item $n = 427$\\ \vspace{.2in}
\end{enumerate}

You might find that the results aren't as clear here.  This is because you really need several random integers $a$ here (and that isn't a good exercise to do by hand).\\

\subsection{SageMath} SageMath has a built-in primality tester.  The syntax is:
\begin{lstlisting}
	2 in Primes()
\end{lstlisting}
and the output will either be True or False (in the example, True, since 2 is prime).

Go back to your First Explorations examples and determine which one is prime using SageMath:

\begin{enumerate}[1.]
	\item 292
	\item 293
	\item 299
	\item 313
	\item 427
	\item 757
	\item 759
	\item 290829273
\end{enumerate}


\end{document}