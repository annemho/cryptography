\documentclass[12pt]{amsart}
\pagestyle{plain}

\usepackage{amsthm, setspace, framed, url, hyperref, lscape}
\usepackage{enumerate}

\usepackage[none]{hyphenat}
\usepackage[pdftex]{graphicx}
\usepackage{enumerate}
\usepackage{tikz}
\usepackage{courier}
\makeatletter
\def\@settitle{\begin{center}%
  \baselineskip14\p@\relax
  \bfseries
  \uppercasenonmath\@title
  \@title
  \ifx\@subtitle\@empty\else
     \\[1ex]\uppercasenonmath\@subtitle
     \footnotesize\mdseries\@subtitle
  \fi
  \end{center}%
}
\def\subtitle#1{\gdef\@subtitle{#1}}
\def\@subtitle{}
\makeatother

\setlength{\textwidth}{6.0in}
\setlength{\textheight}{8.8in}
\setlength{\oddsidemargin}{0.25in}
\setlength{\evensidemargin}{0.25in}
\setlength{\topmargin}{0in}

\theoremstyle{plain}
\newtheorem{thm}{Theorem}[section]
\newtheorem{lem}[thm]{Lemma}
\newtheorem*{cor}{Corollary}
\newtheorem{quest}{Question}

\theoremstyle{definition}
\newtheorem*{defn}{Definition}
\newtheorem*{ex}{Example}

\theoremstyle{remark}
\newtheorem*{rem}{Remark}
\newtheorem*{note}{Note}
\newtheorem{case}{Case}

\newcommand{\R}{\mathbb{R}}
\newcommand{\Z}{\mathbb{Z}}
\newcommand{\C}{\mathbb{C}}
\newcommand{\N}{\mathbb{N}}
\newcommand{\QQ}{\mathbb{Q}}
\newcommand{\Rnn}{\R^{n\times n}} 
\newcommand{\Rn}{\R^{n}} 

\newcommand{\bx}{{\bf x}}
\newcommand{\bv}{{\bf v}}
\newcommand{\bw}{{\bf w}}
\newcommand{\bu}{{\bf u}}

\newcommand{\bit}{\begin{itemize}}
\newcommand{\eit}{\end{itemize}}
\newcommand{\ben}{\begin{enumerate}}
\newcommand{\een}{\end{enumerate}}
\newcommand{\bea}{\begin{eqnarray*}}
\newcommand{\eea}{\end{eqnarray*}}
\newcommand{\bpf}{\begin{proof}}
\newcommand{\epf}{\end{proof}\ms}
\newcommand{\bthm}{\begin{thm}}
\newcommand{\ethm}{\end{thm}}
\newcommand{\bdefn}{\begin{defn}}
\newcommand{\edefn}{\end{defn}}
\newcommand{\bex}{\begin{ex}}
\newcommand{\eex}{\end{ex}}
\newcommand{\bde}{\begin{description}}
\newcommand{\ede}{\end{description}}
\newcommand{\bcen}{\begin{center}}
\newcommand{\ecen}{\end{center}}
\newcommand{\bq}{\begin{quest}}
\newcommand{\eq}{\end{quest}}

\newcommand*\circled[1]{\tikz[baseline=(char.base)]{
            \node[shape=circle,draw,inner sep=2pt] (char) {#1};}}

\begin{document}

\onehalfspacing

\title[]{Cryptography Handout 09}
\subtitle{Congruences}
\maketitle

\section{Properties of Congruences}
\begin{thm}
Let $a,b,c,n$ be integers with $n \neq 0$.
\begin{enumerate}[1.]
	\item $a \equiv 0 \bmod n$ if and only if $n \mid a$.
	\item $a \equiv a \bmod n$.
	\item $a \equiv b \bmod n$ if and only if $b \equiv a \bmod n$.
	\item If $a \equiv b$ and $b \equiv c \bmod n$, then $a \equiv c \bmod n$.
\end{enumerate}
\end{thm}

\noindent Write the proof for property 3, keeping in mind you must show the if and only if statement:
\vspace{2.5in}

\begin{thm}
Let $a,b,c,d,n$ be integers with $n \neq 0$, and suppose $a \equiv b \bmod n$ and $c \equiv d \bmod n$.  Then:
\begin{itemize}
	\item $a + c \equiv b + d (\bmod n)$,
	\item $a - c \equiv b - d (\bmod n)$, and
	\item $ac \equiv bd (\bmod n)$.
\end{itemize}
\end{thm}

\begin{note}
This basically tells us that we have addition, subtraction, and multiplication operations which behave the way we expect.
\end{note}

\newpage \bex
Addition and Multiplication with $\mathbb{Z}_5$:\\
\begin{center}
\begin{tabular}{c|p{.3in}p{.3in}p{.3in}p{.3in}p{.3in}}
+ & 0 & 1 & 2 & 3 & 4\\ \hline
0 & & & & &\vspace{.2in}\\ 
1 & & & & &\vspace{.2in}\\
2 & & & & &\vspace{.2in}\\
3 & & & & &\vspace{.2in}\\
4 & & & & &\vspace{.2in}\\
\end{tabular}
\begin{tabular}{c|p{.3in}p{.3in}p{.3in}p{.3in}p{.3in}}
$\times$ & 0 & 1 & 2 & 3 & 4\\ \hline
0 & & & & &\vspace{.2in}\\ 
1 & & & & &\vspace{.2in}\\
2 & & & & &\vspace{.2in}\\
3 & & & & &\vspace{.2in}\\
4 & & & & &\vspace{.2in}\\
\end{tabular}
\end{center}
\eex

\bex
Addition and Multiplication with $\mathbb{Z}_4$:\\
\begin{center}
\begin{tabular}{c|p{.3in}p{.3in}p{.3in}p{.3in}}
+ & 0 & 1 & 2 & 3 \\ \hline
0 & & & &\vspace{.2in}\\ 
1 & & & &\vspace{.2in}\\
2 & & & &\vspace{.2in}\\
3 & & & &\vspace{.2in}\\
\end{tabular}
\begin{tabular}{c|p{.3in}p{.3in}p{.3in}p{.3in}}
$\times$ & 0 & 1 & 2 & 3 \\ \hline
0 & & & &\vspace{.2in}\\ 
1 & & & &\vspace{.2in}\\
2 & & & &\vspace{.2in}\\
3 & & & &\vspace{.2in}\\
\end{tabular}
\end{center}
\eex

\noindent \textbf{Question:} Do you notice any patterns or differences between  $\mathbb{Z}_5$ and  $\mathbb{Z}_4$?\\
\vspace{2in}

\newpage \section{Division and Inverses}

\noindent \textbf{Recall:} This was a problem from Mission 2: the ciphertext \texttt{GEZXDS} was encrypted by a Hill cipher with a $2 \times 2$ matrix.  The plaintext is \texttt{solved}.  Find the encryption matrix $M$.
	
	Problems:
	\vspace{4in}
	
\noindent \textbf{Question:} How do we know when we have a multiplicative inverse or not?  (We'll figure it out.)

	\begin{enumerate}[1.]
	\item Go back to your $\mathbb{Z}_5$ and $\mathbb{Z}_4$ multiplication tables.  Which elements have inverses?
	\vspace{1.5in}	
	
	\newpage \item Compute the $\gcd$ of all of these elements with inverses with $n$ in $\mathbb{Z}_n$.\\ \vspace{2in}
	
	\item Form a conjecture about which elements have inverses.  If you're not sure, write out the multiplication tables for $\mathbb{Z}_6$ and $\mathbb{Z}_7$ too, and do the same steps as above.\\ \vspace{2.5in}
	
	\item Given what you have conjectured, which elements should have inverses in $\mathbb{Z}_{20}$?  Which elements do not have inverses?
\end{enumerate}



\end{document}