\documentclass[12pt]{amsart}
\pagestyle{plain}

\usepackage{amsthm, setspace, framed, url, hyperref, lscape}
\usepackage[none]{hyphenat}
\usepackage[pdftex]{graphicx}
\usepackage{enumerate}
\usepackage{tikz}
\usepackage{courier}
\makeatletter
\def\@settitle{\begin{center}%
  \baselineskip14\p@\relax
  \bfseries
  \uppercasenonmath\@title
  \@title
  \ifx\@subtitle\@empty\else
     \\[1ex]\uppercasenonmath\@subtitle
     \footnotesize\mdseries\@subtitle
  \fi
  \end{center}%
}
\def\subtitle#1{\gdef\@subtitle{#1}}
\def\@subtitle{}
\makeatother

\setlength{\textwidth}{6.0in}
\setlength{\textheight}{8.8in}
\setlength{\oddsidemargin}{0.25in}
\setlength{\evensidemargin}{0.25in}
\setlength{\topmargin}{0in}

\theoremstyle{plain}
\newtheorem{thm}{Theorem}[section]
\newtheorem{lem}[thm]{Lemma}
\newtheorem*{cor}{Corollary}
\newtheorem{quest}{Question}

\theoremstyle{definition}
\newtheorem*{defn}{Definition}
\newtheorem*{ex}{Example}

\theoremstyle{remark}
\newtheorem*{rem}{Remark}
\newtheorem*{note}{Note}
\newtheorem{case}{Case}

\newcommand{\R}{\mathbb{R}}
\newcommand{\Z}{\mathbb{Z}}
\newcommand{\C}{\mathbb{C}}
\newcommand{\N}{\mathbb{N}}
\newcommand{\QQ}{\mathbb{Q}}
\newcommand{\Rnn}{\R^{n\times n}} 
\newcommand{\Rn}{\R^{n}} 

\newcommand{\bx}{{\bf x}}
\newcommand{\bv}{{\bf v}}
\newcommand{\bw}{{\bf w}}
\newcommand{\bu}{{\bf u}}

\newcommand{\bit}{\begin{itemize}}
\newcommand{\eit}{\end{itemize}}
\newcommand{\ben}{\begin{enumerate}}
\newcommand{\een}{\end{enumerate}}
\newcommand{\bea}{\begin{eqnarray*}}
\newcommand{\eea}{\end{eqnarray*}}
\newcommand{\bpf}{\begin{proof}}
\newcommand{\epf}{\end{proof}\ms}
\newcommand{\bthm}{\begin{thm}}
\newcommand{\ethm}{\end{thm}}
\newcommand{\bdefn}{\begin{defn}}
\newcommand{\edefn}{\end{defn}}
\newcommand{\bex}{\begin{ex}}
\newcommand{\eex}{\end{ex}}
\newcommand{\bde}{\begin{description}}
\newcommand{\ede}{\end{description}}
\newcommand{\bcen}{\begin{center}}
\newcommand{\ecen}{\end{center}}
\newcommand{\bq}{\begin{quest}}
\newcommand{\eq}{\end{quest}}

\newcommand*\circled[1]{\tikz[baseline=(char.base)]{
            \node[shape=circle,draw,inner sep=2pt] (char) {#1};}}

\begin{document}

\onehalfspacing

\title[]{Cryptography Handout 05}
\subtitle{Block Ciphers}
\maketitle

\section{Matrix Facts}
\begin{itemize}
	\item The \textit{determinant} of a $2 \times 2$ matrix $M = \left( \begin{array}{rr}
	a&b\\c&d\\
	\end{array} \right)$ is $\det(M) = ad-bc$.
	\item The \textit{inverse} of a matrix $M$ is denoted $M^{-1}$ and is the one in which $MM^{-1} = M^{-1}M = I$, where $I$ is the identity matrix.  For a $2 \times 2$ matrix $M = \left( \begin{array}{rr}
	a&b\\c&d\\
	\end{array} \right)$, the inverse is
	$M^{-1} = \displaystyle \frac{1}{ad-bc}\left( \begin{array}{rr}
	d&-b\\-c&a\\
	\end{array} \right)$.
\end{itemize}

\vspace{.5in}

\section{Hill Cipher}
\begin{enumerate}[1.]
	\item Choose an $n \times n$ matrix $M$.
	\item Break the plaintext into vectors of length $n$ (using $a = 0, b = 1, \cdots, z = 25$).
	\item To encrypt: multiply each vector by $M$ and reduce $\bmod 26$.
	\item To decrypt: use multiplication with $M^{-1}$.
\end{enumerate}

\bex Suppose we know that $n = 2$ and the following plaintext and ciphertext correspondence:\\
\begin{center}
\begin{tabular}{ll}
plaintext & \texttt{howareyoutoday}\\
CIPHERTEXT & \texttt{ZWSENIUSPLJVEU}\\
\end{tabular}
\end{center}
\eex

\vspace{.5in}

\section{Properties of Good Cryptosystems (Claude Shannon)}
\begin{itemize}
	\item \textbf{Diffusion}: if we change a character of the plaintext, then several characters of the ciphertext change too (and vice versa).
	\item \textbf{Confusion}: the key isn't related to the ciphertext in an easy way, and each character of the ciphertext should depend on several parts of the key.
\end{itemize}



\end{document}