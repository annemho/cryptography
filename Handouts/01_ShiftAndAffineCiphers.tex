\documentclass[12pt]{amsart}
\pagestyle{plain}

\usepackage{amsthm, setspace, framed}
\usepackage[none]{hyphenat}
\usepackage[pdftex]{graphicx}
\usepackage{enumerate}
\usepackage{courier}
\makeatletter
\def\@settitle{\begin{center}%
  \baselineskip14\p@\relax
  \bfseries
  \uppercasenonmath\@title
  \@title
  \ifx\@subtitle\@empty\else
     \\[1ex]\uppercasenonmath\@subtitle
     \footnotesize\mdseries\@subtitle
  \fi
  \end{center}%
}
\def\subtitle#1{\gdef\@subtitle{#1}}
\def\@subtitle{}
\makeatother

\setlength{\textwidth}{6.0in}
\setlength{\textheight}{8.8in}
\setlength{\oddsidemargin}{0.25in}
\setlength{\evensidemargin}{0.25in}
\setlength{\topmargin}{0in}

\theoremstyle{plain}
\newtheorem{thm}{Theorem}[section]
\newtheorem{lem}[thm]{Lemma}
\newtheorem*{cor}{Corollary}
\newtheorem{quest}{Question}

\theoremstyle{definition}
\newtheorem*{defn}{Definition}
\newtheorem*{ex}{Example}

\theoremstyle{remark}
\newtheorem*{rem}{Remark}
\newtheorem*{note}{Note}
\newtheorem{case}{Case}

\newcommand{\R}{\mathbb{R}}
\newcommand{\Z}{\mathbb{Z}}
\newcommand{\C}{\mathbb{C}}
\newcommand{\N}{\mathbb{N}}
\newcommand{\QQ}{\mathbb{Q}}
\newcommand{\Rnn}{\R^{n\times n}} 
\newcommand{\Rn}{\R^{n}} 

\newcommand{\bx}{{\bf x}}
\newcommand{\bv}{{\bf v}}
\newcommand{\bw}{{\bf w}}
\newcommand{\bu}{{\bf u}}

\newcommand{\bit}{\begin{itemize}}
\newcommand{\eit}{\end{itemize}}
\newcommand{\ben}{\begin{enumerate}}
\newcommand{\een}{\end{enumerate}}
\newcommand{\bea}{\begin{eqnarray*}}
\newcommand{\eea}{\end{eqnarray*}}
\newcommand{\bpf}{\begin{proof}}
\newcommand{\epf}{\end{proof}\ms}
\newcommand{\bthm}{\begin{thm}}
\newcommand{\ethm}{\end{thm}}
\newcommand{\bdefn}{\begin{defn}}
\newcommand{\edefn}{\end{defn}}
\newcommand{\bex}{\begin{ex}}
\newcommand{\eex}{\end{ex}}
\newcommand{\bde}{\begin{description}}
\newcommand{\ede}{\end{description}}
\newcommand{\bcen}{\begin{center}}
\newcommand{\ecen}{\end{center}}
\newcommand{\bq}{\begin{quest}}
\newcommand{\eq}{\end{quest}}

\begin{document}

\onehalfspacing

\title[]{Cryptography Handout 01}
\subtitle{Shift and Affine Ciphers}
\maketitle

\section{Shift (Caesar) Cipher}

Assign each letter a number:

\bcen
\begin{tabular}{|p{.25in}|p{.25in}|p{.25in}|p{.25in}|p{.25in}|p{.25in}|p{.25in}|p{.25in}|p{.25in}|p{.25in}|p{.25in}|p{.25in}|p{.25in}|}\hline
a & b & c & d & e & f & g & h & i & j & k & l & m\\
0 & 1 & 2 & 3 & 4 & 5 & 6 & 7 & 8 & 9 & 10 & 11 & 12 \\ \hline
n & o & p & q & r & s & t & u & v & w & x & y & z\\
13 &14 & 15 & 16 & 17 & 18 & 19 & 20 & 21 & 22 & 23 & 24 & 25\\ \hline
\end{tabular}
\ecen

\bex
(Caesar's cipher) Shift each letter by three places.

\bcen
\begin{tabular}{|p{.25in}|p{.25in}|p{.25in}|p{.25in}|p{.25in}|p{.25in}|p{.25in}|p{.25in}|p{.25in}|p{.25in}|p{.25in}|p{.25in}|p{.25in}|}\hline
a & b & c & d & e & f & g & h & i & j & k & l & m\\
\texttt{D} & \texttt{E} & \texttt{F} & \texttt{G} & \texttt{H} & \texttt{I} & \texttt{J} & \texttt{K} & \texttt{L} & \texttt{M} & \texttt{N} & \texttt{O} & \texttt{P} \\ \hline
n & o & p & q & r & s & t & u & v & w & x & y & z\\
\texttt{Q} & \texttt{R} & \texttt{S} & \texttt{T} & \texttt{U} & \texttt{V} & \texttt{W} & \texttt{X} & \texttt{Y} & \texttt{Z} & \texttt{A} & \texttt{B} & \texttt{C}\\ \hline
\end{tabular}
\ecen


\bit
	\item We can think of encryption as $x \mapsto x + \kappa (\bmod 26)$ where $\kappa = 3$.\\
	Plaintext: ``caesar"\\
	\texttt{CIPHERTEXT}: \\
	\item Decryption is $x \mapsto x - \kappa (\bmod 26)$ where $\kappa = 3$.
\eit
\eex

\bex Encrypt ``conquer" using $\kappa = 6$.
%\bcen
%\begin{tabular}{|p{.25in}|p{.25in}|p{.25in}|p{.25in}|p{.25in}|p{.25in}|p{.25in}|p{.25in}|p{.25in}|p{.25in}|p{.25in}|p{.25in}|p{.25in}|}\hline
%a & b & c & d & e & f & g & h & i & j & k & l & m\\
%&&&&&&&&&&&& \\ \hline
%n & o & p & q & r & s & t & u & v & w & x & y & z\\
%&&&&&&&&&&&& \\ \hline
%\end{tabular}
%\ecen
\vspace{1in}
\eex


\bex Decrypt ``\texttt{GPISTEXVE}" using $\kappa = 4$.
\vspace{1in}
\eex


\newpage \section{Affine Ciphers}
\bex
Use $x \mapsto 9 x+ 2  (\bmod 26)$.

\bit
	\item We can think of encryption as $x \mapsto \alpha x + \beta (\bmod 26)$ where $\alpha = 9$ and $\beta = 2$.\\
	Plaintext: ``caesar"\\
	\texttt{CIPHERTEXT}: \\
	\item Decryption uses the inverse function:
	\begin{align*}
	9x + 2 &= y\\
	9x &= y-2\\
	x &= \frac{1}{9}(y-2)\\
	x &= 9^{-1}(y-2)\\
	x &= 3(y-2)\\
	x &= 3y-6\\
	x &= 3y+20
	\end{align*}
	\item Check the decryption:
	\vspace{1in}
\eit
\eex

\bex
Encrypt ``rome" using $x \mapsto 3x+1$ and then find the inverse function.
\eex


	
\end{document}