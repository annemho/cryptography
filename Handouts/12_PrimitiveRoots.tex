\documentclass[12pt]{amsart}
\pagestyle{plain}

\usepackage{amsthm, setspace, framed, url, hyperref, lscape}
\usepackage{enumerate}

\usepackage[none]{hyphenat}
\usepackage[pdftex]{graphicx}
\usepackage{enumerate}
\usepackage{tikz}
\usepackage{courier}
\makeatletter
\def\@settitle{\begin{center}%
  \baselineskip14\p@\relax
  \bfseries
  \uppercasenonmath\@title
  \@title
  \ifx\@subtitle\@empty\else
     \\[1ex]\uppercasenonmath\@subtitle
     \footnotesize\mdseries\@subtitle
  \fi
  \end{center}%
}
\def\subtitle#1{\gdef\@subtitle{#1}}
\def\@subtitle{}
\makeatother

\setlength{\textwidth}{6.0in}
\setlength{\textheight}{8.8in}
\setlength{\oddsidemargin}{0.25in}
\setlength{\evensidemargin}{0.25in}
\setlength{\topmargin}{0in}

\theoremstyle{plain}
\newtheorem{thm}{Theorem}
\newtheorem{lem}[thm]{Lemma}
\newtheorem*{cor}{Corollary}
\newtheorem{quest}{Question}
\theoremstyle{definition}
\newtheorem*{defn}{Definition}
\newtheorem*{ex}{Example}
\theoremstyle{remark}
\newtheorem*{rem}{Remark}
\newtheorem*{note}{Note}
\newtheorem{case}{Case}

\begin{document}

\onehalfspacing

\title[]{Cryptography Handout 12}
\subtitle{Primitive Roots}
\maketitle

\begin{center}
\emph{Based on \underline{Number Theory Through Inquiry} (Marshall, Odell, and Starbird)}.
\end{center}

\section{Review}

\noindent\textbf{Fermat's Little Theorem.} Let $p$ be a prime number and let $a$ be an integer such that $\gcd(a,p) = 1$. Then $a^{p-1} \equiv 1 \mod p$.

\noindent\textbf{Euler's Theorem.} Let $a$ and $n$ be integers with $n > 0$ such that $\gcd(a,n) = 1$. Then $a^{\varphi(n)} \equiv 1 \mod n$.

\section{Primitive Roots}
\begin{defn}
Let $p$ be a prime.  An integer $g$ such that $\textup{ord}_p(g) = p-1$ is called a \emph{primitive root modulo} $p$.
\end{defn}

\begin{thm}
Let $p$ be a prime and suppose $g$ is a primitive root modulo $p$.  Then the set $\{0,g,g^2,g^3, \cdots, g^{p-1}\}$ forms a complete residue system modulo $p$.
\end{thm}

\begin{quest}
For each of the primes $p$ less than 20, find a primitive root and make a chart showing what powers of the primitive root gives each of the natural numbers less than $p$.  Note any observations.
\end{quest}

\newpage
You might observe the following:
\begin{thm}
Every prime $p$ has a primitive root.
\end{thm}

This is another example of an existence theorem.

\begin{quest}
Consider the prime $p = 13$.  For each divisor $d = 1, 2, 3, 4, 6, 12$ of $12 = p-1$, mark which of the natural numbers in the set $\{1,2,3,4,5,6,7,8,9,10,11,12\}$ have order $d$.
\end{quest}

\newpage
You might have observed that there are $\varphi(d)$ numbers of order $d$ for each $d$.  So in the case of 12, we have
$$\varphi(1) + \varphi(2) + \varphi(3) + \varphi(4) + \varphi(6) + \varphi(12) = 12 = \sum_{d \mid 12} \varphi(d) = 12.$$

In general, the more compact way of writing this is $$\sum_{d \mid n} \varphi(d)$$ which means the sum of the Euler-$\varphi$ function of the natural number divisors of the natural number $n$.  There is a more general relationship between the Euler-$\varphi$ function and divisors, which we'll explore next.

\section{Euler-$\varphi$ and the sums of divisors}
\begin{quest}
Compute the following sums, and make any conjectures based on the patterns you notice.  (In particular, notice which numbers $n$ are primes, powers of primes, or products of primes).
\begin{enumerate}[1.]
	\item $\displaystyle \sum_{d \mid 6} \varphi(d)$\\ \vspace{1.5in}
	\item $\displaystyle \sum_{d \mid 7} \varphi(d)$\\ \vspace{1.5in}
	\newpage\item $\displaystyle \sum_{d \mid 24} \varphi(d)$\\ \vspace{2in}
	\item $\displaystyle \sum_{d \mid 36} \varphi(d)$\\ \vspace{2in}
	\item $\displaystyle \sum_{d \mid 27} \varphi(d)$\\ \vspace{2in}
\end{enumerate}
\end{quest}

\newpage It turns out that we have a series of theorems based on these:
\begin{lem}
If $p$ is a prime, then $\displaystyle \sum_{d \mid p} \varphi(d) = p$.
\end{lem}
\begin{lem}
If $p$ is a prime, then $\displaystyle \sum_{d \mid {p^k}} \varphi(d) = p^k$.
\end{lem}
\begin{lem}
If $p$ and $q$ are two different primes, then $\displaystyle \sum_{d \mid pq} \varphi(d) = pq$.
\end{lem}
\begin{thm}
If $n$ is a natural number, then $\displaystyle \sum_{d \mid n} \varphi(d) = n$.
\end{thm}
Using the previous theorem, we can prove the following statement:
\begin{thm}
Every prime $p$ has $\varphi(p-1)$ primitive roots.
\end{thm}

\begin{ex}
\end{ex}

\end{document}