\documentclass[12pt]{amsart}
\pagestyle{plain}

\usepackage{amsthm, setspace, framed, url, hyperref, lscape}
\usepackage{enumerate}

\usepackage[none]{hyphenat}
\usepackage[pdftex]{graphicx}
\usepackage{enumerate}
\usepackage{tikz}
\usepackage{courier}
\makeatletter
\def\@settitle{\begin{center}%
  \baselineskip14\p@\relax
  \bfseries
  \uppercasenonmath\@title
  \@title
  \ifx\@subtitle\@empty\else
     \\[1ex]\uppercasenonmath\@subtitle
     \footnotesize\mdseries\@subtitle
  \fi
  \end{center}%
}
\def\subtitle#1{\gdef\@subtitle{#1}}
\def\@subtitle{}
\makeatother

\setlength{\textwidth}{6.0in}
\setlength{\textheight}{8.8in}
\setlength{\oddsidemargin}{0.25in}
\setlength{\evensidemargin}{0.25in}
\setlength{\topmargin}{0in}

\theoremstyle{plain}
\newtheorem{thm}{Theorem}
\newtheorem{lem}[thm]{Lemma}
\newtheorem*{cor}{Corollary}
\newtheorem{quest}{Question}
\theoremstyle{definition}
\newtheorem*{defn}{Definition}
\newtheorem*{ex}{Example}
\theoremstyle{remark}
\newtheorem*{rem}{Remark}
\newtheorem*{note}{Note}
\newtheorem{case}{Case}

\begin{document}

\onehalfspacing

\title[]{Cryptography Handout 13}
\subtitle{Square Roots and Squares}
\maketitle

\begin{center}
\emph{Based on \underline{Number Theory Through Inquiry} (Marshall, Odell, and Starbird)}.
\end{center}

\begin{quest} 
Determine which of the numbers $1, 2, 3, \cdots, 12$ are perfect squares modulo 13.  For each such square, list the number or numbers in the set whose square is that number (i.e. its square roots).\\
\begin{center}
\begin{tabular}{|c|p{1.5in}|p{2.5in}|} \hline
Number & Square $\bmod 13$? & Square Root(s)?\\ \hline
1&Yes&$12^2 \bmod 13 \equiv 1 \bmod 13$\\ [.2in]
2&&\\ [.2in]
3&&\\ [.2in]
4&&\\ [.2in]
5&&\\ [.2in]
6&&\\ [.2in]
7&&\\ [.2in]
8&&\\ [.2in]
9&&\\ [.2in]
10&&\\ [.2in]
11&&\\ [.2in]
12&&\\  [.2in]\hline
\end{tabular}
\end{center}

How many numbers (out of the 12) show up as squares $\bmod 13$?
\end{quest}

\newpage
\begin{defn} If $a$ is an integer, $p$ is a prime, and $a \equiv b^2 \mod p$ for some integer $b$, then $a$ is called a \textbf{quadratic residue modulo $p$}.  If $a$ is not congruent to any square modulo $p$, then $a$ is a \textbf{quadratic non-residue modulo $p$}.
\end{defn}

\begin{thm}
Let $p$ be a prime.  Half the numbers not congruent to $0 \bmod p$ in a complete residue system $\bmod p$ are quadratic residues and half are quadratic non-residues.
\end{thm}

\begin{defn} For an odd prime $p$ and a natural number $a$ with $p$ not dividing $a$, the \textbf{Legendre symbol $\left( \frac{a}{p}\right)$} is defined to be:
\[ \left( \frac{a}{p}\right) = \begin{cases} 
      1 & a \text{ is a quadratic residue } \bmod p \\
      -1 & a \text{ is a quadratic non-residue } \bmod p \\
   \end{cases}
\]
\end{defn}

\begin{thm}
Suppose $p$ is an odd prime and $p$ does not divide the numbers $a$ or $b$.  Then $$\left( \frac{ab}{p}\right) = \left( \frac{a}{p}\right)\left( \frac{b}{p}\right).$$
\end{thm}

Rewrite the above theorem in your own words, so you remember what it means:\\ \vspace{1in}

\noindent \textbf{Euler's Criterion (Theorem).}
Suppose $p$ is an odd prime and $p$ does not divide the natural number $a$.  Then $a$ is a quadratic residue $\bmod p$ if and only if $a^{(p-1)/2} \equiv 1 \bmod p$, and $a$ is a quadratic non-residue $\bmod p$ if and only if $a^{(p-1)/2} \equiv -1 \bmod p$.

\newpage \begin{quest} 
Fill out the following table for $p = 7$.
\begin{center}
\begin{tabular}{|c|p{2in}|p{2in}|} \hline
$a$ & $a^{(p-1)/2}$ & $a^{(p-1)/2} \bmod 7$\\ \hline
1&1&1\\ [.2in]
2&&\\ [.2in]
3&&\\ [.2in]
4&&\\ [.2in]
5&&\\ [.2in]
6&&\\ [.2in]
12&&\\  [.2in]\hline
\end{tabular}
\end{center}
You should notice that you only have $1$ and $-1 \bmod p$ in the second column.
\end{quest}

\begin{note}
$1$ is always a quadratic residue.  You might wonder about other numbers too.  Let's start by looking at $-1$:
\end{note}

\begin{thm}
Let $p$ be an odd prime.  Then $-1$ is a quadratic residue $\bmod p$ if and only if
\[ \left( \frac{-1}{p}\right) = \begin{cases} 
      1 & p \equiv 1 \bmod 4 \\
      -1 & p \equiv 3 \bmod 4 \\
   \end{cases}
\]
\end{thm}

\begin{thm}
Let $p$ be an odd prime.  Then
\[ \left( \frac{2}{p}\right) = \begin{cases} 
      1 & p \equiv 1 \text{ or } 7 \bmod 8\\
      -1 & p \equiv 3  \text{ or }  5 \bmod 8\\
   \end{cases}
\]
\end{thm}

Rather than looking at $ \left( \frac{3}{p}\right),  \left( \frac{4}{p}\right), \cdots$, we'll consider $\left( \frac{p}{q}\right)$ for primes $p$ and $q$.

\newpage \begin{quest} 
Fill out the following table assuming that the columns are $p$ and the rows are $q$ (ignore the boxes with $x$).  You should only have $1$ and $-1$.
\begin{center}
\begin{tabular}{|p{.4in}||p{.4in}|p{.4in}|p{.4in}|p{.4in}|p{.4in}|p{.4in}|p{.4in}|p{.4in}|p{.4in}|p{.4in}|} \hline
&3&5&7&11&13&17&19&23&29&31\\ \hline\hline
3 &x&&&&&&&&&\\[.2in] \hline
5 &&x&&&&&&&&\\[.2in] \hline
7 &&&x&&&&&&&\\[.2in] \hline
11 &&&&x&&&&&&\\[.2in] \hline
13 &&&&&x&&&&&\\[.2in] \hline
17 &&&&&&x&&&&\\[.2in] \hline
19 &&&&&&&x&&&\\[.2in] \hline
23 &&&&&&&&x&&\\[.2in] \hline
29 &&&&&&&&&x&\\[.2in] \hline
31 &&&&&&&&&&x\\[.2in] \hline
\end{tabular}
\end{center}
\end{quest}

\begin{quest} 
Using the table you made, make a conjecture about the relationship between $\left( \frac{p}{q}\right)$ and $\left( \frac{q}{p}\right)$.\\ \vspace{1in}
\end{quest}


\end{document}