\documentclass[12pt]{amsart}
\pagestyle{plain}

\usepackage{amsthm, setspace, framed, hyperref}
\usepackage[pdftex]{graphicx}
\usepackage{enumerate}
\usepackage[none]{hyphenat}

\usepackage[left=1in, right=1in, top=1in, bottom=1in]{geometry}
\setlength\parindent{0pt}

\theoremstyle{plain}
\newtheorem{thm}{Theorem}[section]
\newtheorem{lem}[thm]{Lemma}
\newtheorem*{cor}{Corollary}
\newtheorem{quest}{Question}
\theoremstyle{definition}
\newtheorem*{defn}{Definition}
\newtheorem*{ex}{Example}

\begin{document}
\title[]{Cryptography Priority Mission 02}
\begin{tabular*}{\textwidth}{@{\extracolsep{\fill}}l l}
MATH/CSCI 408  & Name: \rule{7cm}{0.5pt} \\
\hline\hline
\end{tabular*} \\
\maketitle

\begin{center}\textbf{Deadline: Thursday, 27 October 2016 at 11:59pm}\\
\end{center}


\section{Rules for Take-Home Exam}
\begin{enumerate}[1.]
	\item You may use the following resources:
		\begin{itemize}
			\item Anne
			\item your notes
			\item the textbook (probably less helpful on this exam)
			\item handouts from class (\textbf{especially the Number Theory summary!})
			\item SageMath code and any SageMath or Python documentation
			\item solutions to previous missions
		\end{itemize}
	\item You may NOT use the following:
		\begin{itemize}
			\item Internet solutions
			\item classmates or other professors
			\item any other source that isn't listed above
		\end{itemize}
	\item You may submit solutions electronically or by paper.
	\item Be sure to show all work, and provide all code.
	\item \textbf{Choose 5 out of the 6 problems}, and clearly denote which one is not going to be graded.  Each problem is worth 20 points for a total of 100 points.
\end{enumerate}


\section{Problems}
Archer, Lana, and Ray get ambushed while on a mission after getting hit by tranquilizer darts in the jungle (from the henchman of the evil villain known as the ``Annihilator.")\\

\begin{enumerate}[1.]
	\item Ray wakes up first but is feeling woozy and has hallucinations in which Mallory is demanding that he computes the following Legendre symbols.  Help him out so that he can focus on walking back to the escape vehicle (be sure to justify your answers).
		\begin{enumerate}[a.]
			\item $\left(\frac{137}{151}\right)$
			\item $\left(\frac{151}{137}\right)$
			\item $\left(\frac{271}{151}\right)$
			\item $\left(\frac{151}{271}\right)$\\
		\end{enumerate}
	\item Lana wakes up in a cell to find that the Annihilator standing before her.  It turns out that the Annihilator is actually an evil mathematician, and she asks Lana to prove that there are infinitely many prime numbers.  Help Lana out by following these steps:
		\begin{enumerate}[a.]
			\item Prove that for any natural number $n$, that $\gcd(n,n+1) = 1$.  [Hint: start by supposing that an integer $a$ is such that $a \mid n$ and $a \mid (n+1)$.  What does that mean in terms of the definitions?  Then, notice that $1 = n+1-n.$  How can you rewrite this using the definitions you just wrote?]\\
			\item Now suppose there are only finitely many primes $p_1, p_2, \cdots, p_N$ for a natural number $N$ (for the sake of contradiction).  Consider the number which is the product of all the primes plus one, or $p_1p_2\cdots p_N+1$.  What can you say using part a. of this problem?\\
		\end{enumerate}
	\item Archer also wakes up in a cell but finds several of the Annihilator's henchmen demanding that he uses the Euclidean algorithm to find the greatest common divisor of 16891 and 589.  Krieger (who is still in communication with Archer via a hidden ear piece) tells Archer that the answer is 19, but the henchmen aren't satisfied until Archer shows all of his work.  Help him out.\\
	
	\item Pam and Cheryl are waiting on the escape vehicle, wondering what is taking so long.  They get bored and start computing the following:
		\begin{enumerate}[a.]
			\item $\varphi(23)$\\
			\item $\varphi(32)$\\
			\item $\varphi(p)$ where $p$ is a prime number\\
			\item $8^{278} \bmod 13$ (be sure to explain how you use Fermat's Little Theorem)\\
			\item $7^{496} \bmod 32$ (be sure to explain how you use Euler's Theorem)\\
		\end{enumerate}
	\item Out of nowhere, Babou jumps out and attacks the Annihilator.  Do the following while Lana and Archer escape:\\
		\begin{enumerate}[a.]
			\item Find a primitive root of $p = 17$, and show you you verify that it is a primitive root.\\
			\item Write a SageMath program to check if an integer $n$ is a primitive root of a given prime $p$.  Be sure to provide an example of $n$ being a primitive root and an example where it isn't.\\
		\end{enumerate}
	\item Now that the Annihilator is defeated (via ocelot), her henchmen want to divide up her treasure (in the form of $x$ gold bricks that she had in a vault).  149 henchmen start to divide up the treasure evenly, but realize that there are 3 gold bars left over, so they get in a fight.  Two henchmen are critically wounded and taken to the hospital.\\
	\begin{enumerate}[a.]
	\item Before the remaining 147 henchmen get in a fight again, you jump in because you believe a peaceful solution is the best solution.  Explain to the henchmen how you can use the Chinese Remainder Theorem to guarantee that the $x$ gold bricks can be divided evenly as long (as there are enough gold bricks).\\
	\item Use SageMath's \texttt{crt(a,b,m,n)} to show what the minimum number of gold bricks $x$ must be.\\
	\end{enumerate}
\end{enumerate}

	
\end{document}
