\documentclass[12pt]{amsart}
\pagestyle{plain}

\usepackage{amsthm, setspace, framed, hyperref}
\usepackage[pdftex]{graphicx}
\usepackage{enumerate}
\usepackage[none]{hyphenat}

\usepackage[left=1in, right=1in, top=1in, bottom=1in]{geometry}
\setlength\parindent{0pt}

\theoremstyle{plain}
\newtheorem{thm}{Theorem}[section]
\newtheorem{lem}[thm]{Lemma}
\newtheorem*{cor}{Corollary}
\newtheorem{quest}{Question}
\theoremstyle{definition}
\newtheorem*{defn}{Definition}
\newtheorem*{ex}{Example}

\begin{document}
\title[]{Cryptography Priority Mission 01 Solutions}
\begin{tabular*}{\textwidth}{@{\extracolsep{\fill}}l l}
MATH/CSCI 408  & Name: \rule{7cm}{0.5pt} \\
\hline\hline
\end{tabular*} \\
\maketitle

\begin{center}\textbf{Deadline: Thursday, 29 September 2016 at 11:59pm}\\
\end{center}


\section{Problems}


\begin{enumerate}[1.]
	\item (Affine cipher) The following ciphertext used the affine cipher $x \mapsto 9x+13$:\\ \texttt{NRXOHLROKVRNKCHAHIXRJASXXITYNZXAAJCTCHKKXO}. Decrypt it.\\
	
	\texttt{A medium dry martini, lemon peel.  Shaken, not stirred.} (James Bond's drink.)\\
	\begin{framed}
	\includegraphics[height=3.4in]{01.jpg}
	\end{framed}
	
	\item (Affine cipher) Suppose you encrypt using an affine cipher, then encrypt the encryption using another affine cipher (both are working $\bmod 26$).  Is there any advantage to doing this?  Why or why not?  Explain in a paragraph, and give a specific, detailed example.\\
	
		There is no advantage to using two affine ciphers since two affine ciphers composed together just yield another affine cipher.   For example, $f_1(x) = x+2$ and$ f_2(x) = 2x+3$.  The two composed together give $(f_1 \circ f_2)(x) = (2x+3)+2 = 2x+5$, which is just another affine cipher.  This means the same frequency analyses and methods can break it.	\\
	
	\item (Hill Cipher)  Barry captures Archer's Hill cipher machine, which uses a $2 \times 2$ matrix $M \bmod 26$.  He realizes that the plaintext \texttt{ba} encrypts to \texttt{HC}, and the plaintext \texttt{zz} encrypts to \texttt{GT}.  What is the matrix $M$?\\
		
	We have:
	\begin{align*}
	\texttt{b} =1 &\leftrightarrow \texttt{H}=7\\
	\texttt{a} =0  &\leftrightarrow \texttt{C}=2\\
	\texttt{z} =25  &\leftrightarrow \texttt{G}=6\\
	\texttt{z} =25  &\leftrightarrow \texttt{T}=19\\
	\end{align*}	
	
	The matrix setup is:
	\begin{align*}
	\left(\begin{array}{cc}
	1 & 0\\
	25 & 25
	\end{array}\right)
	\left(\begin{array}{cc}
	a & b\\
	c & d
	\end{array}\right) &=
	\left(\begin{array}{cc}
	7 & 2\\
	6 & 19
	\end{array}\right)\\
	\left(\begin{array}{cc}
	a & b\\
	c & d
	\end{array}\right) &=
	\left(\begin{array}{cc}
	1 & 0\\
	25 & 25
	\end{array}\right)^{-1}
	\left(\begin{array}{cc}
	7 & 2\\
	6 & 19
	\end{array}\right)\\
	&= \left(\begin{array}{cc}
	1 & 0\\
	25 & 25
	\end{array}\right)
	\left(\begin{array}{cc}
	7 & 2\\
	6 & 19
	\end{array}\right)\\
	&= \left(\begin{array}{cc}
	7 & 2\\
	13 & 5
	\end{array}\right)
	\end{align*}
	
	\item (Hill Cipher) Suppose you are given the following ciphertext:\\
	\texttt{ESIZEHAXPDILHJDTBQEHSJZXXHQFIBKZJYWUQWEDKDEUDMHJTWPVQLEHHMMFKBMUZXEUZTESIZYL}\\
	Given the encryption matrix:
	$$\left(\begin{array}{rr}
	1 & 5\\
	2 & 3\\
	\end{array}\right)$$
	Decrypt the message.\\

	The inverse of the matrix is$$\left(\begin{array}{rr}
	7 & 23\\
	4 & 11\\
	\end{array}\right)$$
	
	We can use SageMath to decrypt:\\
	
	\noindent \texttt{We are not now that strength which in old days}\\
	\texttt{Moved earth and heaven; that which we are, we are.}\\
	(M quoting Tennyson in \emph{Skyfall}.)\\
	
	\begin{framed}
	\includegraphics[height=3.4in]{04.jpg}
	\end{framed}
	
	 \item (Proof) Write a proof for the following statements:\\
		\begin{enumerate}[a.]
			\item Let $a,b,n$ be integers.  If $\gcd(a,n) = 1$ and $\gcd(b,n) = 1$, then $\gcd(ab,n) = 1$.\\
				
				\begin{proof}
				Suppose $a,b$ and $n$ are integers.  Assume $\gcd(a,n) = 1$ and $\gcd(b,n) = 1$.  Then by the theorem in class, we know that there exist integers $x,y,z,w$ in which
				\begin{align*}
				ax+ny &= 1\\
				bz+nw &=1
				\end{align*}
				Consider 
				\begin{align*}
				(ax+ny)(bz+nw) &= 1\\
				abxz+anxw+bnyz+n^2yz &= 1\\
				ab(xz) + n(axs+byz+nyw) &=1
				\end{align*}
				Using the same theorem, this means that $\gcd(ab,n) = 1$.
				\end{proof}
				
			\item Let $a,b,c,$ and $n$ be integers with $n > 0$.  If $a \equiv b \mod n$ and  $b \equiv c \mod n$, then\\ $a \equiv c \mod n$.\\
			
				\begin{proof}
				Suppose $a,b,c,n$ are integers with $n > 0$.  Assume $a \equiv b \mod n$ and  $b \equiv c \mod n$.  This means $n \mid (a-b)$ and $n \mid (b-c)$.  In other words, $a-b = nk$ for some integer $k$ and $b-c = nl$ for some integers $l$.
				Consider
				\begin{align*}
				(a-b)+(b-c) &= nk + nl\\
				a-c &= n(k+l)
				\end{align*}
				So $n \mid (a-c)$ or $a \equiv c \mod n$.
				\end{proof}
				
		\end{enumerate}
	 \item (SageMath) We are using a new alphabet $\{A,B,C,D,E,F,G\}$ (perhaps corresponding to musical notes).  Associate the letters with the numbers $\{0,1,2,3,4,5,6\}$, respectively.\\
		\begin{enumerate}[a.]
			\item Using the shift cipher with a shift of 5, encrypt the following sequence of notes for Twinkle Twinkle Little Star: \texttt{ccggaagffeeddcggffeedggffeedccggaagffeeddc}.\\
			
			A shift by 5 yields: \texttt{aaeeffeddccbbaddccbbaddccbbaaaeeffeddccbba}	.\\
			
			\item Write a program that performs affine ciphers on the musical alphabet.  Provide the code as well as an example of output.\\

			See an example below or run the file: \begin{verbatim} Exam01_Problem06b.sagews
			\end{verbatim}
			
			\begin{framed}
			\includegraphics[height=3.4in]{06b.jpg}
			\end{framed}
			
		\end{enumerate}
\end{enumerate}

	
\end{document}
