\documentclass[12pt]{amsart}
\pagestyle{plain}

\usepackage{amsthm, setspace, framed, hyperref}
\usepackage[pdftex]{graphicx}
\usepackage{enumerate}
\usepackage[none]{hyphenat}

\usepackage[left=1in, right=1in, top=1in, bottom=1in]{geometry}
\setlength\parindent{0pt}

\theoremstyle{plain}
\newtheorem{thm}{Theorem}[section]
\newtheorem{lem}[thm]{Lemma}
\newtheorem*{cor}{Corollary}
\newtheorem{quest}{Question}
\theoremstyle{definition}
\newtheorem*{defn}{Definition}
\newtheorem*{ex}{Example}

\begin{document}
\title[]{Cryptography Priority Mission 01}
\begin{tabular*}{\textwidth}{@{\extracolsep{\fill}}l l}
MATH/CSCI 408  & Name: \rule{7cm}{0.5pt} \\
\hline\hline
\end{tabular*} \\
\maketitle

\begin{center}\textbf{Deadline: Thursday, 29 September 2016 at 11:59pm}\\
\end{center}


\section{Rules for Take-Home Exam}
\begin{enumerate}[1.]
	\item You may use the following resources:
		\begin{itemize}
			\item Anne
			\item your notes
			\item the textbook
			\item handouts from class
			\item SageMath code and any SageMath or Python documentation
			\item solutions to previous missions
		\end{itemize}
	\item You may NOT use the following:
		\begin{itemize}
			\item Internet solutions
			\item classmates
			\item other professors
			\item any other source that isn't listed above
		\end{itemize}
	\item You may submit solutions electronically or by paper.
	\item Be sure to show all work, and provide all code.
	\item \textbf{Choose 5 out of the 6 problems}, and clearly denote which one is not going to be graded.  Each problem is worth 20 points for a total of 100 points.
\end{enumerate}


\section{Problems}


\begin{enumerate}[1.]
	\item (Affine cipher) The following ciphertext used the affine cipher $x \mapsto 9x+13$:\\ \texttt{NRXOHLROKVRNKCHAHIXRJASXXITYNZXAAJCTCHKKXO}.\\ Decrypt it.\\
	\item (Affine cipher) Suppose you encrypt using an affine cipher, then encrypt the encryption using another affine cipher (both are working $\bmod 26$).  Is there any advantage to doing this?  Why or why not?  Explain in a paragraph, and give a specific, detailed example.\\
	\item (Hill Cipher)  Barry captures Archer's Hill cipher machine, which uses a $2 \times 2$ matrix $M \bmod 26$.  He realizes that the plaintext \texttt{ba} encrypts to \texttt{HC}, and the plaintext \texttt{zz} encrypts to \texttt{GT}.  What is the matrix $M$?\\
	\item (Hill Cipher) Suppose you are given the following ciphertext:\\
	\texttt{ESIZEHAXPDILHJDTBQEHSJZXXHQFIBKZJYWUQWEDKDEUDMHJTWPVQLEHHMMFKBMUZXEUZTESIZYL}\\
	Given the encryption matrix:
	$$\left(\begin{array}{rr}
	1 & 5\\
	2 & 3\\
	\end{array}\right)$$
	Decrypt the message.\\
	\newpage \item (Proof) Write a proof for the following statements:\\
		\begin{enumerate}[a.]
			\item Let $a,b,n$ be integers.  If $\gcd(a,n) = 1$ and $\gcd(b,n) = 1$, then $\gcd(ab,n) = 1$.\\
			\item Let $a,b,c,$ and $n$ be integers with $n > 0$.  If $a \equiv b \mod n$ and  $b \equiv c \mod n$, then\\ $a \equiv c \mod n$.\\
		\end{enumerate}
	\item (SageMath) We are using a new alphabet $\{A,B,C,D,E,F,G\}$ (perhaps corresponding to musical notes).  Associate the letters with the numbers $\{0,1,2,3,4,5,6\}$, respectively.\\
		\begin{enumerate}[a.]
			\item Using the shift cipher with a shift of 5, encrypt the following sequence of notes for Twinkle Twinkle Little Star: \texttt{ccggaagffeeddcggffeedggffeedccggaagffeeddc}.\\
			\item Write a program that performs affine ciphers on the musical alphabet.  Provide the code as well as an example of output.\\
		\end{enumerate}
\end{enumerate}

	
\end{document}
