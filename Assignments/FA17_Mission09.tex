\documentclass[12pt]{amsart}
\pagestyle{plain}

\usepackage{amsthm, setspace, framed, hyperref}
\usepackage[pdftex]{graphicx}
\usepackage{enumerate}
\usepackage[none]{hyphenat}

\usepackage[left=1in, right=1in, top=1in, bottom=1in]{geometry}
\setlength\parindent{0pt}

\theoremstyle{plain}
\newtheorem{thm}{Theorem}[section]
\newtheorem{lem}[thm]{Lemma}
\newtheorem*{cor}{Corollary}
\newtheorem{quest}{Question}
\theoremstyle{definition}
\newtheorem*{defn}{Definition}
\newtheorem*{ex}{Example}

\begin{document}
\title[]{Cryptography Mission 09}
\begin{tabular*}{\textwidth}{@{\extracolsep{\fill}}l l}
MATH/CSCI 408  & Name: \rule{7cm}{0.5pt} \\
\hline\hline
\end{tabular*} \\
\maketitle

\begin{center}\textbf{Deadline: Thursday, 9 November 2017 at 10:50am}\\

This mission covers Sections 7.1, 7.2, and 7.5
\end{center}

\begin{framed}
Check one:\\

\framebox(12,12){} I received help from the following classmate(s) on this assignment:\\

\rule{15cm}{0.5pt}.\\

\framebox(12,12){} I did not receive any help on this assignment.
\end{framed}


\section{Graded Problems}

\begin{enumerate}[1.]
	\item (Final Project) Provide a document with your brainstorming ideas and logistics (electronic submission or paper submission are both fine).  Some ideas of things to address:
    \begin{itemize}
    	\item What is the format of your presentation?
		\item What is the mathematical content in your project (this is required)?
        \item Who are you going to invite, and how are you going to convince them to show up?  Note: one of the projects involves advertising.
        \item There will be tables, and Dr. Ho can provide poster boards to the groups that had requested them. If you're using a computer, are there enough outlets in the space (go check it out)?  If not, make sure you have your devices charged ahead of time.
        \item For those of you who have thought about prizes or other supplies, who exactly do you need to contact?  How are you going to contact them and by when?
        \item Do you have a clear sense of how you're communicating with your team members?  Have you organized logistics?\\
    \end{itemize}
    \item For each of the following, determine the parity of $x$ in the discrete log problem $$\alpha^x \equiv \beta \bmod p.$$  Then, depending on the parity, write some code (and email it) to do a brute-force search through only the even or only the odd numbers to find all values of $x$ up to the prime $p$.
    	\begin{enumerate}[a.]
        	\item $5^x \equiv 75 \bmod 107$
            \begin{framed}
            \vspace{1.2in}
            \end{framed}
            \item $11^{23} \equiv 98 \bmod 349$
            \begin{framed}
            \vspace{1.2in}
            \end{framed}
        \end{enumerate}
     
    \item Use the Baby Step, Giant Step algorithm to determine $x$ for $4^x \equiv 20 \bmod 61$.
    \begin{framed}
    \vspace{2.5in}
    \end{framed}
    \item ElGamal
    \begin{enumerate}[a.]
    	\item Mallory is sending Sterling Archer a message using ElGamal.  His public key is $(p,\alpha,\beta) = (89,6,31)$.  If Mallory's message is $m = 77$ and she chooses $k = 3$, show the encryption process to determine $r$ and $t$.
        \begin{framed}
        \vspace{1in}
        \end{framed}
        \item What is the computation Archer does to decrypt the message?  Verify that this equals the original message of $m = 77$.
        \begin{framed}
        \vspace{1in}
        \end{framed}
        \item Cheryl and Pam are also sending their own ElGamal message but they use $p = 17$ and $\alpha = 3$.  Pam chooses her secret to be $a = 6$, so $\beta = 15$.  Cheryl sends the ciphertext $(r,t) = (7,6)$.  Determine the plaintext $m$.
        \begin{framed}
        \vspace{1in}
        \end{framed}
    \end{enumerate}
    \item (Honors) When proving the Index Calculus method in class for the Discrete Log problem, we showed that $$\beta \alpha^r = \prod_i p_i^{b_i} \bmod p \text{ implies } L_{\alpha}^{\beta} = -r + \sum_i b_i L_{\alpha}(p_i) \bmod (p-1)$$ where $L_{\alpha}$ denotes the log base $\alpha$ function.  Show the missing steps to get from the first to the second equation. (Hint: use properties of the log function).
	\begin{framed}
    \vspace{3in}
    \end{framed}

\end{enumerate}
	
    
\section{Recommended Exercises}
\noindent These will not be graded but are recommended if you need more practice.
\begin{itemize}
	\item Section 7.6 \# 1, 2, 3, 4, 6, 12
    \item Section 7.7 \# 1, 2, 3
\end{itemize}
\end{document}
