\documentclass[12pt]{amsart}
\pagestyle{plain}

\usepackage{amsthm, setspace, framed, hyperref}
\usepackage[pdftex]{graphicx}
\usepackage{enumerate}
\usepackage[none]{hyphenat}

\usepackage[left=1in, right=1in, top=1in, bottom=1in]{geometry}
\setlength\parindent{0pt}

\theoremstyle{plain}
\newtheorem{thm}{Theorem}[section]
\newtheorem{lem}[thm]{Lemma}
\newtheorem*{cor}{Corollary}
\newtheorem{quest}{Question}
\theoremstyle{definition}
\newtheorem*{defn}{Definition}
\newtheorem*{ex}{Example}

\begin{document}
\title[]{Cryptography Mission 09 Dossier}
\begin{tabular*}{\textwidth}{@{\extracolsep{\fill}}l l}
MATH/CSCI 408  & Name: \rule{7cm}{0.5pt} \\
\hline\hline
\end{tabular*} \\
\maketitle

\begin{center}\textbf{Deadline: Thursday, 17 November 2016 at 3:05pm}\\

This mission covers Sections 6.3, 6.4, and 7.1
\end{center}

\begin{framed}
Check one:\\

\framebox(12,12){} I received help from the following classmate(s) on this assignment:\\

\rule{15cm}{0.5pt}.\\

\framebox(12,12){} I did not receive any help on this assignment.
\end{framed}

\section{Graded Problems}

\begin{enumerate}[1.]
	\item Read through the Miller-Rabin Primality Test (6.3 p. 178).  Work through the example, and write a new example here.\\
		\begin{framed}
		\vspace{1.5in}
		\end{framed}
	\item Use the Fermat Factoring method to factor 70747.\\
		\begin{framed}
		\vspace{2in}
		\end{framed}
	\item Use the $p-1$ Factoring Algorithm to factor 4757.\\ 
		\begin{framed}
		\vspace{3in}
		\end{framed}
	\item Use SageMath's \texttt{factor()} to check your answers to problems 1 and 2.\\
		\begin{framed}
		\vspace{1in}
		\end{framed}
	\item Given $p = 17$.  Solve the following discrete logs problems if possible.  If not, explain why.
		\begin{enumerate}[a.]
			\item $14 \equiv 3^x \mod 17$
			\begin{framed}
			\vspace{2.5in}
			\end{framed}
			\newpage \item $5 \equiv 4^x \mod 17$
			\begin{framed}
			\vspace{2.5in}
			\end{framed}
		\end{enumerate}
\end{enumerate}


\section{Recommended Exercises}
\noindent These will not be graded but are recommended if you need more practice.
\begin{itemize}
	\item Section 6.8: \# 9, 13, 18
	\item Section 6.9: \# 11
\end{itemize}
	
\end{document}
