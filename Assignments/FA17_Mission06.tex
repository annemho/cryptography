\documentclass[12pt]{amsart}
\pagestyle{plain}

\usepackage{amsthm, setspace, framed, hyperref}
\usepackage[pdftex]{graphicx}
\usepackage{enumerate}

\usepackage[left=1in, right=1in, top=1in, bottom=1in]{geometry}
\setlength\parindent{0pt}

\theoremstyle{plain}
\newtheorem{thm}{Theorem}[section]
\newtheorem{lem}[thm]{Lemma}
\newtheorem*{cor}{Corollary}
\newtheorem{quest}{Question}
\theoremstyle{definition}
\newtheorem*{defn}{Definition}
\newtheorem*{ex}{Example}

\begin{document}
\title[]{Cryptography Mission 06}
\begin{tabular*}{\textwidth}{@{\extracolsep{\fill}}l l}
MATH/CSCI 408  & Name: \rule{7cm}{0.5pt} \\
\hline\hline
\end{tabular*} \\
\maketitle

\begin{center}\textbf{Deadline: Thursday, 12 October 2017 at 10:50am}\\

This mission covers Sections 3.7, 3.9, 3.10, 4.1, 4.2.
\end{center}

\section{Graded Problems}

\begin{enumerate}[1.]
\item Write a short \texttt{for} loop program in CoCalc to compute powers of a number $a \bmod p$ where $p$ is a prime. In other words, the input of your code should be $a$ and $p$.  The output should be all powers of $a^i \bmod p$ for $i = 1, 2, \cdots, p-1$.  Email this code to Dr. Ho.\\
\item Using the previous problem, are the following values are primitive roots (i.e. the order is $p-1)$ or not?
	\begin{enumerate}[a.]
	\item $a = 4, p = 23$
    	\begin{framed}
    	\vspace{.3in}
		\end{framed}
    \item $a = 5, p = 47$
    	\begin{framed}
    	\vspace{.3in}
		\end{framed}
    \end{enumerate}
\item Given an integer $a$ and an odd prime $p$.  Determine if $a$ is a quadratic residue mod $p$ or not.  Justify.
	\begin{enumerate}[a.]
		\item $a = 4, p = 11$
		\begin{framed}
		\vspace{.5in}
		\end{framed}
		\item $a = 2, p = 19$
		\begin{framed}
		\vspace{.5in}
		\end{framed}
		\item $a = 3, p = 29$
		\begin{framed}
		\vspace{.5in}
		\end{framed}
	\end{enumerate}
	\item Given an integer $a$ (not congruent to 0 $\mod p$) and an odd prime $p$, recall that the Legendre symbol is defined as: 
	\[ \left( \frac{a}{p}\right) = \begin{cases} 
      1 & a \text{ is a quadratic residue } \bmod p \\
      -1 & a \text{ is a quadratic non-residue } \bmod p \\
   \end{cases}
	\]
	Evaluate the following by using CoCalc's \texttt{kronecker(a,b)} function, which is the same as the Legendre symbol:
	\begin{enumerate}[a.]
		\item $\left( \frac{7}{13} \right)$
		\begin{framed}
		\vspace{.3in}
		\end{framed}
		\item $\left( \frac{7}{19} \right)$
		\begin{framed}
		\vspace{.3in}
		\end{framed}
		\item $\left( \frac{2}{13} \right)$
		\begin{framed}
		\vspace{.3in}
		\end{framed}
		\item $\left( \frac{14}{13} \right)$
		\begin{framed}
		\vspace{1in}
		\end{framed}
	\end{enumerate}
	\item Recall that the Law of Quadratic Reciprocity says:  Let $p$ and $q$ be odd primes.  Then
		\[ \left( \frac{p}{q}\right) = \begin{cases} 
      		\left( \frac{q}{p}\right) & p \equiv 1 \bmod 4 \text{ or } q \equiv 1 \bmod 4\\
      		-\left(\frac{q}{p}\right) & p \equiv q \equiv 3 \bmod 4\\
  		 \end{cases}
		\]
		Compute the following.  Be sure to show all work.
		\begin{enumerate}[a.]
		\item  Compute $\left(\frac{97}{101}\right)$.
		\begin{framed}
		\vspace{.3in}
		\end{framed}
		\item Explain using Quadratic Reciprocity what the value will be for $\left(\frac{101}{97}\right)$.
		\begin{framed}
		\vspace{.5in}
		\end{framed}
		\newpage \item Compute $\left(\frac{3}{107}\right)$
		\begin{framed}
		\vspace{.3in}
		\end{framed}
		\item Explain using Quadratic Reciprocity what the value will be for $\left(\frac{107}{3}\right)$
		\begin{framed}
		\vspace{.5in}
		\end{framed}
		\end{enumerate}
\item Play with the Simplified DES code (written by Dr. N. McNew at Towson): \url{https://tinyurl.com/fa17-crypto-DES}. Specifically, type in a 12-bit input \texttt{100100100100}, a key $K$ = \texttt{111111110}, and 7 rounds.  Write your output here:
		\begin{framed}
		\vspace{.5in}
		\end{framed}
\item (Honors) Read Section 4.7 (Meet-in-the-Middle Attacks) and summarize what you learned in a paragraph or two:
	\begin{framed}
    \vspace{3in}
    \end{framed}



\end{enumerate}
	
\end{document}
