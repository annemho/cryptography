\documentclass[12pt]{amsart}
\pagestyle{plain}

\usepackage{amsthm, setspace, framed, hyperref}
\usepackage[pdftex]{graphicx}
\usepackage{enumerate}

\usepackage[left=1in, right=1in, top=1in, bottom=1in]{geometry}
\setlength\parindent{0pt}

\theoremstyle{plain}
\newtheorem{thm}{Theorem}[section]
\newtheorem{lem}[thm]{Lemma}
\newtheorem*{cor}{Corollary}
\newtheorem{quest}{Question}
\theoremstyle{definition}
\newtheorem*{defn}{Definition}
\newtheorem*{ex}{Example}

\begin{document}
\title[]{Cryptography Mission 01 Solutions}
\begin{tabular*}{\textwidth}{@{\extracolsep{\fill}}l l}
MATH/CSCI 408  & Name: \rule{7cm}{0.5pt} \\
\hline\hline
\end{tabular*} \\
\maketitle

\begin{center}\textbf{Deadline: Thursday, 1 September 2016 at 3:05pm}\\

This mission covers Sections 2.1, 2.2, and 2.3.
\end{center}

\section{Graded Problems}

\begin{enumerate}[1.]
	\item Do the following survey: \url{http://tinyurl.com/16AnneHoSurvey}\\
	\item (T\&W 2.13 \# 4)  Consider an affine cipher ($\bmod 26$).  You do a chosen plaintext attack using \texttt{hahaha}.  The ciphertext is \texttt{NONONO}.  Determine the encryption function.\\ \begin{framed}Notice that we can set up ordered pairs with plaintext and ciphertext, so our pairs are \texttt{(h,N) and \texttt{(a,O)}}, which can be translated into (7,13) and (0,14).  We want to find the line containing both.  The slope is $$m = \frac{14-13}{0-7} = -\frac{1}{7} = -1(7^{-1}) = 25(15) \bmod 26 = 11 \bmod 26.$$
	The general affine encryption function is $f(x) = ax+b$ or $x \mapsto ax+b (\bmod 26)$.  Here, we have $x \mapsto 11 x + b$, so solving for $b$ using one of our points, we get $1 \bmod 26$, and our answer is $x \mapsto 11x+14$.\end{framed}
	 \item This problem involves the Dancing Men code from a Sherlock Holmes story.
		\begin{enumerate}[a.]
			\item Read Section 2.5 (Sherlock Holmes), and describe (in a paragraph) how Sherlock figures out which dancing man represents the letter \texttt{e} as well as the letter \texttt{r}.\\ \begin{framed}Holmes first observes what the flags mean.  Then he realizes that the most frequent symbol is likely \texttt{e}.  Since the English language only has so many words with the structure \texttt{-e-e-} (including words like lever, never, or sever), he could use partial information to conclude that the last letter must be \texttt{r.}  \end{framed}
			\item Explain in one sentence what the little flags mean.\\\begin{framed}The flags denote the ends of words.\end{framed}
			\item Draw the dancing men figures that would correspond to the plaintext: \texttt{math}.\\ \begin{framed}\begin{center}\includegraphics[height=.7in]{DancingMan.jpg}\end{center} \end{framed}
		\end{enumerate}
		\item (T\&W 2.14 \# 2)  The following ciphertext was the output of a shift cipher:
		\begin{center}
			\texttt{LCLLEWLJAZLNNZMVYIYLHRMHZA}
		\end{center}
		By performing a frequency count, guess the key used in the cipher.  What is the decrypted plaintext?\\ \begin{framed}\texttt{L} shows up 6 times, \texttt{Z} shows up 3 times, \texttt{A,H,M,N,Y} show up twice, and \texttt{C,E,I,J,R,V,W} show up once.  Since the most common letter in the English alphabet is \texttt{e}, then we assume that the shift was from \texttt{e} to \texttt{L}, meaning all letters shifted by 6.  This helps us translate the rest of the ciphertext to the plaintext: \texttt{eve expects eggs for breakfast}. \end{framed}
	\item SageMathCloud:\\
		\begin{enumerate}[a.]
			\item Set up a SageMathCloud account (\url{https://cloud.sagemath.com}).  Create a new project (Cryptography).  You can use this project folder to hold all of your SageMath code.  \\
			\item From Moodle (\url{https://moodle.coastal.edu/}), download Dr. Ho's Vigenere cipher code (Vigenere Cipher.sagews) and upload it to your SageMathCloud project folder.\\
			\item Run through the cells of code, and then encode the phrase, ``Hey, Mr. Tambourine Man, play a song for me" by using the key word ``DYLAN".  Write down the ciphertext.  Be sure to print your code or email it to Dr. Ho too.\\ \begin{framed}One version of SageMath code is below:
	\begin{framed}\begin{center}
	\includegraphics[width=5.5in]{Mission01_5.jpg}
	\end{center}\end{framed}
	\end{framed}
		\end{enumerate}
\end{enumerate}

	
\end{document}
