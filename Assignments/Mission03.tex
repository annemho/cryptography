\documentclass[12pt]{amsart}
\pagestyle{plain}

\usepackage{amsthm, setspace, framed, hyperref}
\usepackage[pdftex]{graphicx}
\usepackage{enumerate}

\usepackage[left=1in, right=1in, top=1in, bottom=1in]{geometry}
\setlength\parindent{0pt}

\theoremstyle{plain}
\newtheorem{thm}{Theorem}[section]
\newtheorem{lem}[thm]{Lemma}
\newtheorem*{cor}{Corollary}
\newtheorem{quest}{Question}
\theoremstyle{definition}
\newtheorem*{defn}{Definition}
\newtheorem*{ex}{Example}

\begin{document}
\title[]{Cryptography Mission 03 Dossier}
\begin{tabular*}{\textwidth}{@{\extracolsep{\fill}}l l}
MATH/CSCI 408  & Name: \rule{7cm}{0.5pt} \\
\hline\hline
\end{tabular*} \\
\maketitle

\begin{center}\textbf{Deadline: Thursday, 15 September 2016 at 3:05pm}\\

This mission covers Sections 2.8, 2.9, 2.10, and all Classical Ciphers.
\end{center}

\begin{framed}
Check one:\\

\framebox(12,12){} I received help from the following classmate(s) on this assignment:\\

\rule{15cm}{0.5pt}.\\

\framebox(12,12){} I did not receive any help on this assignment.
\end{framed}


\section{Graded Problems}

\begin{enumerate}[1.]
	\item Read through the ``Examples of basic usage" section for Python's pseudo-random number generators (\url{https://docs.python.org/2/library/random.html}).
	\begin{enumerate}[a.]
		\item In SageMath, generate 5 pseudo-random numbers using \texttt{random()}, and write them here (round to 4 decimal places).
			\begin{framed}
			\vspace{1in}
			\end{framed}
		\item Write down the code for generating a random integer from 1 to 100.  Generate 3 such numbers and write them here.
			\begin{framed}
			\vspace{1in}
			\end{framed}
		\newpage \item Write down the code for generating a random odd from 1 to 101.  Generate 3 such numbers and write them here.
			\begin{framed}
			\vspace{1in}
			\end{framed}
	\end{enumerate}
	\item Bletchley Park was where a lot of cryptography happened during World War II.  Watch \url{https://www.youtube.com/watch?v=wlWVpOzgrL4}, and write down two facts that you learned here.
	\begin{framed}
	\vspace{3in}
	\end{framed}
	\item If \texttt{11010010} is your plaintext message, and \texttt{10101010} is the key, what is the ciphertext using a One-Time Pad? 
	\begin{framed}
	\vspace{2in}
	\end{framed}
	\newpage \item Make a list of all the Classical Ciphers we have covered so far to remember what they are.  These might be used in the ``Escape Room" activity on 9/15.
	\begin{framed}
	\vspace{4in}
	\end{framed}
\end{enumerate}

\section{Recommended Exercises}
\noindent None from the book this time!

	
\end{document}
