\documentclass[12pt]{amsart}
\pagestyle{plain}

\usepackage{amsthm, setspace, framed, hyperref}
\usepackage[pdftex]{graphicx}
\usepackage{enumerate}
\usepackage[none]{hyphenat}

\usepackage[left=1in, right=1in, top=1in, bottom=1in]{geometry}
\setlength\parindent{0pt}

\theoremstyle{plain}
\newtheorem{thm}{Theorem}[section]
\newtheorem{lem}[thm]{Lemma}
\newtheorem*{cor}{Corollary}
\newtheorem{quest}{Question}
\theoremstyle{definition}
\newtheorem*{defn}{Definition}
\newtheorem*{ex}{Example}

\begin{document}
\title[]{Cryptography Mission 10}
\begin{tabular*}{\textwidth}{@{\extracolsep{\fill}}l l}
MATH/CSCI 408  & Name: \rule{7cm}{0.5pt} \\
\hline\hline
\end{tabular*} \\
\maketitle

\begin{center}\textbf{Deadline: Thursday, 16 November 2017 at 10:50am}\\

This mission covers Sections 9.1, 9.2
\end{center}

\section{Graded Problems}

\begin{enumerate}[1.]
	\item (Final Project) Provide a list of legitimate references/sources for your project that you have considered (at least 3).  They might not be included in your final project, but you should be doing thorough background research.
    \begin{framed}
    \vspace{2in}
    \end{framed}
    \item (RSA Signature) Alice has a public key $(e_A, n) = (5, 119)$.
    \begin{enumerate}[a.]
    \item Bob's message is $m = 22$.  Eve (the mail carrier) gives Bob a signed version of the message in which Alice's signature is $y = 71$.  Is the signature valid?  Explain.
    \begin{framed}
   \vspace{1in}
    \end{framed}
    \item Bob finds a second letter on his desk that is signed ``Alice" with $(m,y) = (17,67)$.  Is this letter actually from Alice?  Explain.
    \begin{framed}
    \vspace{1in}
    \end{framed}
    \end{enumerate}
    \item (Modified from T\&W 9.6 \# 4) There are many variations to the ElGamal digital signature scheme that can be obtained by altering the signing equation $s \equiv k^{-1}(m-ar) \bmod (p-1)$.  Here are some variations.
    	\begin{enumerate}[a.]
        	\item Consider the signing equation $s \equiv a^{-1}(m-kr) \bmod (p-1)$.  Show that the verification $\alpha^m \equiv (\alpha^a)^s r^r \bmod p$ is a valid verification procedure.
            \begin{framed}
            \vspace{2in}
            \end{framed}
             \item Consider the signing equation $s \equiv am + kr \bmod (p-1)$.  Show that the verification $\alpha^s \equiv (\alpha^a)^m r^r \bmod p$ is a valid verification procedure.
            \begin{framed}
            \vspace{1.5in}
            \end{framed}
      		\item Create an example (possibly using CoCalc) with $m = 10$ and the prime $p = 11$.  Be sure to show the setup stage, the signing stage, and the verification process.  Email any code to Dr. Ho.
            \begin{framed}
            \vspace{2.5in}
            \end{framed}
        \end{enumerate}
      
    \item (Honors - Modified from T\& W \# 8) Consider the following variation of the ElGamal signature scheme.  Alice chooses a prime $p$ and a primitive root $\alpha$.  She also chooses a function $f(x)$ that, given an integer $x$ with $0 \leq x < p$, returns an integer $f(x)$ with $0 \leq f(x) < p-1$.  For example, $f(x) = x^7-3x+2 \bmod (p-1)$.  She chooses a secret integer $a$ and computes $\beta \equiv \alpha^a \bmod p$.  The numbers $p, \alpha, \beta$ and $f(x)$ are made public.
	\begin{itemize}
    	\item Signing $m$:
        	\begin{enumerate}[1.]
            	\item Alice chooses a random integer $k$ with $\gcd(k,p-1) = 1$.
                \item She computes $r \equiv \alpha^k \bmod p$.
                \item She computes $s \equiv k^{-1}(m-f(r)a) \bmod (p-1)$.  The signed message is $(m,r,s)$.
            \end{enumerate}
        \item Verifying:
        	\begin{enumerate}[1.]
            	\item Bob computes $v_1 \equiv \beta^{f(r)}r^s \bmod p$.
                \item He also computes $v_2 \equiv \alpha^m \bmod p$.
                \item If $v_1 \equiv v_2 \bmod p$, then the signature is valid.
            \end{enumerate}
    \end{itemize}
Show that if all the procedures are followed correctly, then the verification equation is true (Hint: Start by solving $s \equiv k^{-1}(m-f(r)a) \bmod (p-1)$ for $f(r)$.  Then, plug this into $v_1$ and use definitions until you get $v_2$.)
\begin{framed}
\vspace{3in}
\end{framed}
\end{enumerate}
	
    
\section{Recommended Exercises}
\noindent These will not be graded but are recommended if you need more practice.
\begin{itemize}
	\item Section 9.6 \# 1, 2, 5, 6
    \item Section 9.7 \# 1
\end{itemize}
\end{document}
