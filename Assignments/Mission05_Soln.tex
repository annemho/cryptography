\documentclass[12pt]{amsart}
\pagestyle{plain}

\usepackage{amsthm, setspace, framed, hyperref}
\usepackage[pdftex]{graphicx}
\usepackage{enumerate}
\usepackage[none]{hyphenat}

\usepackage[left=1in, right=1in, top=1in, bottom=1in]{geometry}
\setlength\parindent{0pt}

\theoremstyle{plain}
\newtheorem{thm}{Theorem}[section]
\newtheorem{lem}[thm]{Lemma}
\newtheorem*{cor}{Corollary}
\newtheorem{quest}{Question}
\theoremstyle{definition}
\newtheorem*{defn}{Definition}
\newtheorem*{ex}{Example}

\begin{document}
\title[]{Cryptography Mission 05 Solutions}
\begin{tabular*}{\textwidth}{@{\extracolsep{\fill}}l l}
MATH/CSCI 408  & Name: \rule{7cm}{0.5pt} \\
\hline\hline
\end{tabular*} \\
\maketitle

\begin{center}\textbf{Deadline: Thursday, 6 October 2016 at 3:05pm}\\

This mission covers Sections 3.4, 3.6, and current issues.
\end{center}

\section{Graded Problems}

\begin{enumerate}[1.]
	\item (Fermat's Little Theorem and Euler's Theorem) Compute each of the following without the aid of a calculator or computer (you can double-check with some code though).
		\begin{enumerate}[a.]
			\item $\varphi(35)$
			\begin{framed}
			24
			\end{framed}
			\item $514^{372} \mod 13$
			\begin{framed}
			$(514^{12})^{31} \mod 13 \equiv 1^{31} \mod 13 \equiv 1 \mod 13$
			\end{framed}
			\item $2^{49} \mod 15$
			\begin{framed}
			$(2^{8})^6)(2) \mod 15 \equiv 1^6(2) \mod 15 \equiv 2 \mod 15$
			\end{framed}
		\end{enumerate}
\end{enumerate}


	
\end{document}
