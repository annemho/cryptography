\documentclass[12pt]{amsart}
\pagestyle{plain}

\usepackage{amsthm, setspace, framed, hyperref}
\usepackage[pdftex]{graphicx}
\usepackage{enumerate}

\usepackage[left=1in, right=1in, top=1in, bottom=1in]{geometry}
\setlength\parindent{0pt}

\theoremstyle{plain}
\newtheorem{thm}{Theorem}[section]
\newtheorem{lem}[thm]{Lemma}
\newtheorem*{cor}{Corollary}
\newtheorem{quest}{Question}
\theoremstyle{definition}
\newtheorem*{defn}{Definition}
\newtheorem*{ex}{Example}

\begin{document}
\title[]{Cryptography Mission 04 Dossier}
\begin{tabular*}{\textwidth}{@{\extracolsep{\fill}}l l}
MATH/CSCI 408  & Name: \rule{7cm}{0.5pt} \\
\hline\hline
\end{tabular*} \\
\maketitle

\begin{center}\textbf{Deadline: Thursday, 14 September 2017 at 10:50am}\\

This mission covers Sections 2.7 and 2.9.
\end{center}

\begin{framed}
Check one:\\

\framebox(12,12){} I received help from the following classmate(s) on this assignment:\\

\rule{15cm}{0.5pt}.\\

\framebox(12,12){} I did not receive any help on this assignment.
\end{framed}

\section{Graded Problems}

\begin{enumerate}[1.]
\item (T \& W 2.13 \# 14) The ciphertext \texttt{GEZXDS} was encrypted by a Hill cipher with a $2 \times 2$ matrix.  The plaintext is \texttt{solved}.  Find the encryption matrix $M$.
	\begin{framed}
	\vspace{4.7in}
	\end{framed}
		\item (T \& W 2.13 \# 16)
		\begin{enumerate}[a.]
		\item The ciphertext \texttt{ELNI} was encrypted by a Hill cipher with a $2 \times 2$ matrix.  The plaintext is \texttt{dont}.  Find the encryption matrix $M$.
		\begin{framed}
		\vspace{3.4in}
		\end{framed}
		\item Suppose the ciphertext is \texttt{ELNK} and the plaintext is still \texttt{dont}.  Find the encryption matrix.  Note that the second column of the matrix is changed.  This shows that the entire second column of the encryption matrix is involved in obtaining the last character of the ciphertext.
		\begin{framed}
		\vspace{3.4in}
		\end{framed}
		\end{enumerate}
	\item Read through the ``Examples of basic usage" section for Python's pseudo-random number generators (\url{https://docs.python.org/2/library/random.html}).
	\begin{enumerate}[a.]
		\item In SageMath, generate 5 pseudo-random numbers using \texttt{random()}, and write them here (round to 4 decimal places).
			\begin{framed}
			\vspace{.5in}
			\end{framed}
		\item Write down the code for generating a random integer from 1 to 100.  Generate 3 such numbers and write them here.
			\begin{framed}
			\vspace{.5in}
			\end{framed}
		 \item Write down the code for generating a random odd from 1 to 101.  Generate 3 such numbers and write them here.
			\begin{framed}
			\vspace{.5in}
			\end{framed}
	\end{enumerate}
	\item Bletchley Park was where a lot of cryptography happened during World War II.  Watch \url{https://www.youtube.com/watch?v=wlWVpOzgrL4}, and write down two facts that you learned here.
	\begin{framed}
	\vspace{1.5in}
	\end{framed}
	\item If \texttt{11010010} is your plaintext message, and \texttt{10101010} is the key, what is the ciphertext using a One-Time Pad? 
	\begin{framed}
	\vspace{1in}
	\end{framed}
	 \item (Honors) Let $a,b,c,d,e,f$ be integers $\bmod 26$.  Consider the following combination of the Hill and affine ciphers: represent a block of plaintext as a pair $(x,y) \bmod 26$.  The corresponding ciphertext $(u,v)$ is 
	 $$(x,y)\left(\begin{array}{ll}
	 a&b\\
	 c&d\\
	\end{array}\right)+(e,f) \equiv (u,v) \bmod 26.$$  
	Encrypt the plaintext $\texttt{carolina}$ using the values below:
	$$(x,y)\left(\begin{array}{ll}
	 3&4\\
	 3&1\\
	\end{array}\right)+(8,11) \equiv (u,v) \bmod 26$$  
	\begin{framed}
	\vspace{5in}
	\end{framed}
\end{enumerate}

\section{Recommended Exercises}
\noindent These will not be graded but are recommended if you need more practice.
\begin{itemize}
	\item Section 2.13: \# 13, 15, 17, 19
	\item Section 2.14: \# 9
\end{itemize}
	
\end{document}
