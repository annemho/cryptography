\documentclass[12pt]{amsart}
\pagestyle{plain}

\usepackage{amsthm, setspace, framed, hyperref}
\usepackage[pdftex]{graphicx}
\usepackage{enumerate}
\usepackage[none]{hyphenat}

\usepackage[left=1in, right=1in, top=1in, bottom=1in]{geometry}
\setlength\parindent{0pt}

\theoremstyle{plain}
\newtheorem{thm}{Theorem}[section]
\newtheorem{lem}[thm]{Lemma}
\newtheorem*{cor}{Corollary}
\newtheorem{quest}{Question}
\theoremstyle{definition}
\newtheorem*{defn}{Definition}
\newtheorem*{ex}{Example}

\begin{document}
\title[]{Cryptography Mission 07 Dossier}
\begin{tabular*}{\textwidth}{@{\extracolsep{\fill}}l l}
MATH/CSCI 408  & Name: \rule{7cm}{0.5pt} \\
\hline\hline
\end{tabular*} \\
\maketitle

\begin{center}\textbf{Deadline: Thursday, 19 October 2017 at 10:50am}\\

This mission covers Sections 4.2, 4.4, and 4.8.
\end{center}

\begin{framed}
Check one:\\

\framebox(12,12){} I received help from the following classmate(s) on this assignment:\\

\rule{15cm}{0.5pt}.\\

\framebox(12,12){} I did not receive any help on this assignment.
\end{framed}


\section{Graded Problems}

\begin{enumerate}[1.]
	\item This problem will walk you through a couple of steps of the DES that are different from the Simplified DES model.  Read the DES section in the textbook (Section 4.4--skip 4.4.1).
	\begin{enumerate}[a.]
		\item In a couple of sentences and with an example, explain what the Initial Permutation step does.
		\begin{framed}
		\vspace{1in}
		\end{framed}
		\item In a couple of sentences, explain how the keys $K_1, K_2, \cdots, K_{16}$ are generated given a key $K$.
		\begin{framed}
		\vspace{1in}
		\end{framed}
		\item If $B_1 = 101010$, explain how you would use the first S-box $S_1$ to get an output.
		\begin{framed}
		\vspace{1in}
		\end{framed}
	\end{enumerate}
	\item Read the password security section in the book (Section 4.8).
		\begin{enumerate}[a.]
		\item Explain in a sentence or two what \textbf{salt} means in this context.  Provide an example:
		\begin{framed}
		\vspace{1.5in}
		\end{framed}
		 \item Based on the lecture (or the ``DES, AES, and Passwords" PDF on Moodle), explain why \texttt{GreatPassword123} is a bad password.  Be sure to explain what kind of attack might be used to crack this password easily.
		\begin{framed}
		\vspace{2in}
		\end{framed}
		\end{enumerate}
	\item For a bit string $S$, let $\bar{S}$ denote the complement of the string by changing all 1s to 0s and 0s to 1s (equivalently, this can be defined as $\bar{S} = S \oplus 1111\cdots$).  Show with an explicit example that if the simplified DES key $K$ encrypts a plaintext $P$ to a ciphertext $C$, then $\bar{K}$ encrypts $\bar{P}$ to $\bar{C}$.  You can use the code  \url{https://tinyurl.com/fa17-crypto-DES} again.
	\begin{framed}
	\vspace{2in}
	\end{framed}
	\newpage \item (Honors) Consider the following DES-like encryption method: Start with a 6-bit message.  Divide it into two blocks of length 3 (a left half and a right half): $M_0M_1$.  The key $K$ consists of 3 bits.  One round of encryption starts with a pair $M_jM_{j+1}$ and the output is the pair $M_{j+1}M_{j+2}$ where $M_{j+2} = M_j \oplus K$ (where the operation is exclusive or, aka addition $\bmod 2$).  This is done for $m$ rounds, so the ciphertext is $M_mM_{m+1}$.
	\begin{enumerate}[a.]
		\item Suppose the initial input is $000111$ and the key is $K = 101$.  What is the ciphertext $M_3M_4$?
		\begin{framed}
		\vspace{2in}
		\end{framed}
		\item If you have a machine that does the $m$-round encryption, how would you use the same machine to decrypt the ciphertext $M_{m}M_{m+1}$ (with the same key $K$)?  Show this explicitly with the example from part (a).
	\end{enumerate}
	\begin{framed}
	\vspace{2in}
	\end{framed}
\end{enumerate}
	

\section{Recommended Exercises}
\noindent These will not be graded but are recommended if you need more practice.
\begin{itemize}
	\item Section 4.9: \# 2, 7
\end{itemize}
	
\end{document}
