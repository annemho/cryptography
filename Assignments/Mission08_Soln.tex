\documentclass[12pt]{amsart}
\pagestyle{plain}

\usepackage{amsthm, setspace, framed, hyperref}
\usepackage[pdftex]{graphicx}
\usepackage{enumerate}
\usepackage[none]{hyphenat}

\usepackage[left=1in, right=1in, top=1in, bottom=1in]{geometry}
\setlength\parindent{0pt}

\theoremstyle{plain}
\newtheorem{thm}{Theorem}[section]
\newtheorem{lem}[thm]{Lemma}
\newtheorem*{cor}{Corollary}
\newtheorem{quest}{Question}
\theoremstyle{definition}
\newtheorem*{defn}{Definition}
\newtheorem*{ex}{Example}

\begin{document}
\title[]{Cryptography Mission 08 Solutions}
\begin{tabular*}{\textwidth}{@{\extracolsep{\fill}}l l}
MATH/CSCI 408  & Name: \rule{7cm}{0.5pt} \\
\hline\hline
\end{tabular*} \\
\maketitle

\begin{center}\textbf{Deadline: Thursday, 10 November 2016 at 3:05pm}\\

This mission covers Sections 6.1 and 6.2.
\end{center}

\section{Graded Problems}

\begin{enumerate}[1.]
	\item Work through the RSA code (RSA.sagews) on Moodle.  
	\begin{enumerate}[a.]
		\item Notice that if you run the code multiple times, you will end up getting different encrypted text.  In a sentence or two, explain why this is:
		\begin{framed}
		Random numbers are generated as part of the algorithm.
		\end{framed}
		\item Encrypt a one-line phrase, and email the input and output to me (please don't write this one by hand!).
	\end{enumerate}
	 \item Part of the RSA lectures was a claim that we can factor $n = pq$ by just knowing $n$ and $\varphi(n)$ (see notes on using a certain polynomial and the quadratic formula).  Write out the details of how you would do this for $n = 27679$.  You can use SageMath for the Euler phi function, but you cannot use it for direct factoring here.
		\begin{framed}
		\begin{align*}
		X^2-(n-\varphi(n)+1)X+n &\Rightarrow X^2-400X+27679\\
		X &= \frac{400 \pm \sqrt{(-400)^2-4(1)(27679)}}{2}\\
		&= 311, 89
		\end{align*}
		\end{framed}
	\item Part of the discussion on RSA attacks was a mention of \textbf{continued fractions}.
		\begin{enumerate}[a.]
			\item Read the intro, motivation and notation, and basic formula sections on Wikipedia's continued fractions page: \url{https://en.wikipedia.org/wiki/Continued_fraction}.\\
			\item Write an example of a finite continued fraction here.\\
				\begin{framed}
				$$\frac{26}{7} = 3+\frac{1}{1+\frac{1}{2+\frac{1}{2}}}$$
				\end{framed}
			\item Explain in a sentence which types of numbers would have an infinite continued fraction.
				\begin{framed}
				An irrational number would yield an infinite continued fraction.
				\end{framed}
		\end{enumerate}
\end{enumerate}

	
\end{document}
