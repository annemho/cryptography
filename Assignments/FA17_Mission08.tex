\documentclass[12pt]{amsart}
\pagestyle{plain}

\usepackage{amsthm, setspace, framed, hyperref}
\usepackage[pdftex]{graphicx}
\usepackage{enumerate}
\usepackage[none]{hyphenat}

\usepackage[left=1in, right=1in, top=1in, bottom=1in]{geometry}
\setlength\parindent{0pt}

\theoremstyle{plain}
\newtheorem{thm}{Theorem}[section]
\newtheorem{lem}[thm]{Lemma}
\newtheorem*{cor}{Corollary}
\newtheorem{quest}{Question}
\theoremstyle{definition}
\newtheorem*{defn}{Definition}
\newtheorem*{ex}{Example}

\begin{document}
\title[]{Cryptography Mission 08}
\begin{tabular*}{\textwidth}{@{\extracolsep{\fill}}l l}
MATH/CSCI 408  & Name: \rule{7cm}{0.5pt} \\
\hline\hline
\end{tabular*} \\
\maketitle

\begin{center}\textbf{Deadline: Thursday, 2 November 2017 at 10:50am}\\

This mission covers Sections 6.1, 6.2, 6.3, and 6.4.
\end{center}

\begin{framed}
Check one:\\

\framebox(12,12){} I received help from the following classmate(s) on this assignment:\\

\rule{15cm}{0.5pt}.\\

\framebox(12,12){} I did not receive any help on this assignment.
\end{framed}


\section{Graded Problems}

\begin{enumerate}[1.]
	\item (Final Project) Write down a short paragraph summary of your project and any team members here.  Remember that Dr. Ho assigned you a general topic, but you can choose your own team or work individually within that group.
    \begin{framed}
    \vspace{2.5in}
    \end{framed}
    \item Work through the RSA code here: \url{https://tinyurl.com/fa17-crypto-rsa}.  Notice that if you run the code multiple times, you will end up getting different encrypted text.
    \begin{enumerate}[a.]
    \item In a sentence or two, explain why this is:
   \begin{framed}
	\vspace{.5in}
	\end{framed}
    \item Encrypt a one-line phrase, and email the input and output to Dr. Ho (please don't write this one by hand!).
	\end{enumerate}
    \item Part of the RSA lectures was a claim that we can factor $n = pq$ by just knowing $n$ and $\varphi(n)$.  We do this by setting up the quadratic equation $X^2 - (n - \varphi(n) + 1)X + n$ and solving for its roots.  Write out the details of how you would do this for $n = 27679$.  You can use CoCalc for the Euler phi function, but show the details of the rest of your work.
		\begin{framed}
		\vspace{2in}
		\end{framed}
   \item Read through the Miller-Rabin Primality Test (6.3 p. 178).  Work through the example.  Then, use the primality test for $n = 101$.
		\begin{framed}
		\vspace{2in}
		\end{framed}
	\item Use the Fermat Factoring method to factor 70747.
		\begin{framed}
		\vspace{2in}
		\end{framed}
	\item Use the $p-1$ Factoring Algorithm to factor 4757. You can use CoCalc to help with the computations.
		\begin{framed}
		\vspace{3in}
		\end{framed}
	\item (Honors) Read Sections 6.5 (The RSA Challenge) and 6.6 (An Application to Treaty Verification).  Summarize both sections in a short paragraph each.
    	\begin{framed}
		\vspace{3in}
		\end{framed}
\end{enumerate}

\section{Recommended Exercises}
\noindent These will not be graded but are recommended if you need more practice.
\begin{itemize}
	\item Section 6.8: \# 1, 3, 5, 7, 13, 19
    \item Section 6.9 \# 1, 2, 5, 7
\end{itemize}
	
\end{document}
