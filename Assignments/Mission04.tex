\documentclass[12pt]{amsart}
\pagestyle{plain}

\usepackage{amsthm, setspace, framed, hyperref}
\usepackage[pdftex]{graphicx}
\usepackage{enumerate}

\usepackage[left=1in, right=1in, top=1in, bottom=1in]{geometry}
\setlength\parindent{0pt}

\theoremstyle{plain}
\newtheorem{thm}{Theorem}[section]
\newtheorem{lem}[thm]{Lemma}
\newtheorem*{cor}{Corollary}
\newtheorem{quest}{Question}
\theoremstyle{definition}
\newtheorem*{defn}{Definition}
\newtheorem*{ex}{Example}

\begin{document}
\title[]{Cryptography Mission 04 Dossier}
\begin{tabular*}{\textwidth}{@{\extracolsep{\fill}}l l}
MATH/CSCI 408  & Name: \rule{7cm}{0.5pt} \\
\hline\hline
\end{tabular*} \\
\maketitle

\begin{center}\textbf{Deadline: Thursday, 22 September 2016 at 3:05pm}\\

This mission covers Sections 3.1 and 3.3.
\end{center}

\begin{framed}
Check one:\\

\framebox(12,12){} I received help from the following classmate(s) on this assignment:\\

\rule{15cm}{0.5pt}.\\

\framebox(12,12){} I did not receive any help on this assignment.
\end{framed}

\section{Graded Problems}

\begin{enumerate}[1.]
	\item Let $F_1 = 1, F_2 = 1, F_{n+1} = F_n + F_{n-1}$ define the Fibonacci numbers $1,1,2,3,5, \cdots$.
		\begin{enumerate}[a.]
			\item List the first 15 Fibonacci numbers.
				\begin{framed}
				\vspace{2in}
				\end{framed}
			\item Compute the greatest common divisor for the following pairs: $F_{10}$ and $F_7$, $F_6$ and $F_9$, $F_6$ and $F_{12}$, $F_{10}$ and $F_{13}$.
				\begin{framed}
				\vspace{1.8in}
				\end{framed}
			\item Look at your previous examples. It turns out that $\gcd(F_m,F_n) = F_{\gcd(m,n)}$.  Write out \textbf{two} specific and detailed examples to verify that you believe this is true.
			\begin{framed}
				\vspace{1.3in}
			\end{framed}
			\begin{framed}
				\vspace{1.3in}
			\end{framed}
			\item Play with some examples, and make a conjecture about $\gcd(F_n,F_{n-1})$ for $n \geq 1$.  Are there any patterns?  Describe them here.
			\begin{framed}
				\vspace{2in}
			\end{framed}
		\end{enumerate}
	\item You can compute a gcd using SageMath's \texttt{gcd(a,b)}.  Determine the solution for the following gcd computations.
		\begin{enumerate}[a.]
			\item $\gcd(234,6013)$
			\begin{framed}
			\vspace{1in}
			\end{framed}
			\newpage\item $\gcd(74951,26269)$
			\begin{framed}
			\vspace{1in}
			\end{framed}
			\item $\gcd(5223389,188434513)$
			\begin{framed}
			\vspace{1in}
			\end{framed}
		\end{enumerate}
	\item In class, we started practicing writing proofs or formal mathematical arguments.  In this problem, we're going to walk through the proof of a theorem.
	\begin{enumerate}[a.]
		\item The Theorem you want to prove is: Let $a,b,c,d,$ and $n$ be integers with $n > 0$.  If $a \equiv b \bmod n$ and $c \equiv d \bmod n$, then $ac \equiv bd \mod n$.  First, come up with an example (with specific numbers) to convince yourself this is true.
		\begin{framed}
		\vspace{2in}
		\end{framed}
		\item Which part of the theorem is the hypothesis?  This is what you assume.
		\begin{framed}
		\vspace{1.5in}
		\end{framed}
		\newpage \item Which part of the theorem is the conclusion? This will be what you show is true based on the hypothesis.
		\begin{framed}
		\vspace{1in}
		\end{framed}
		\item Write out the definition of $a \equiv b \bmod n$.
		\begin{framed}
		\vspace{1in}
		\end{framed}
		\item Now write the proof.  Start by assuming the hypothesis.  Use the necessary definitions and work your way towards the conclusion.
		\begin{framed}
		\vspace{3.5in}
		\end{framed}
	\end{enumerate}
\end{enumerate}

\section{Recommended Exercises}
\noindent These will not be graded but are recommended if you need more practice.
\begin{itemize}
	\item Section 3.13: \# 1, 4, 5, 7
	\item Section 3.14: \# 1
\end{itemize}
	
\end{document}
