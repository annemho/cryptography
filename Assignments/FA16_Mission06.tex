\documentclass[12pt]{amsart}
\pagestyle{plain}

\usepackage{amsthm, setspace, framed, hyperref}
\usepackage[pdftex]{graphicx}
\usepackage{enumerate}
\usepackage[none]{hyphenat}

\usepackage[left=1in, right=1in, top=1in, bottom=1in]{geometry}
\setlength\parindent{0pt}

\theoremstyle{plain}
\newtheorem{thm}{Theorem}[section]
\newtheorem{lem}[thm]{Lemma}
\newtheorem*{cor}{Corollary}
\newtheorem{quest}{Question}
\theoremstyle{definition}
\newtheorem*{defn}{Definition}
\newtheorem*{ex}{Example}

\begin{document}
\title[]{Cryptography Mission 06 Dossier}
\begin{tabular*}{\textwidth}{@{\extracolsep{\fill}}l l}
MATH/CSCI 408  & Name: \rule{7cm}{0.5pt} \\
\hline\hline
\end{tabular*} \\
\maketitle

\begin{center}\textbf{Deadline: Thursday, 20 October 2016 at 3:05pm}\\

This mission covers Sections 3.9, 3.10, and 4.2.
\end{center}

\begin{framed}
Check one:\\

\framebox(12,12){} I received help from the following classmate(s) on this assignment:\\

\rule{15cm}{0.5pt}.\\

\framebox(12,12){} I did not receive any help on this assignment.
\end{framed}

\section{Graded Problems}

\begin{enumerate}[1.]
	\item Given an integer $a$ and an odd prime $p$.  Determine if $x^2 \equiv a \mod p$ has a solution or not.  Justify.
	\begin{enumerate}[a.]
		\item $a = 4, p = 11$
		\begin{framed}
		\vspace{1in}
		\end{framed}
		\item $a = 2, p = 19$
		\begin{framed}
		\vspace{1in}
		\end{framed}
		\item $a = 3, p = 29$
		\begin{framed}
		\vspace{1in}
		\end{framed}
	\end{enumerate}
	\item Given an integer $a$ (not congruent to 0 $\mod p$) and an odd prime $p$, recall that the Legendre symbol is defined as: 
	\[ \left( \frac{a}{p}\right) = \begin{cases} 
      1 & a \text{ is a quadratic residue } \bmod p \\
      -1 & a \text{ is a quadratic non-residue } \bmod p \\
   \end{cases}
	\]
	Evaluate the following:
	\begin{enumerate}[a.]
		\item $\left( \frac{7}{13} \right)$
		\begin{framed}
		\vspace{1.1in}
		\end{framed}
		\item $\left( \frac{7}{19} \right)$
		\begin{framed}
		\vspace{1.1in}
		\end{framed}
		\item $\left( \frac{2}{13} \right)$
		\begin{framed}
		\vspace{1.1in}
		\end{framed}
		\item $\left( \frac{14}{13} \right)$
		\begin{framed}
		\vspace{1.1in}
		\end{framed}
	\end{enumerate}
	\item Recall that the Law of Quadratic Reciprocity says:  Let $p$ and $q$ be odd primes.  Then
		\[ \left( \frac{p}{q}\right) = \begin{cases} 
      		\left( \frac{q}{p}\right) & p \equiv 1 \bmod 4 \text{ or } q \equiv 1 \bmod 4\\
      		\left( -\frac{q}{p}\right) & p \equiv q \equiv 3 \bmod 4\\
  		 \end{cases}
		\]
		Compute the following.  Be sure to show all work.
		\begin{enumerate}[a.]
		\item  $\left(\frac{97}{101}\right)$
		\begin{framed}
		\vspace{1in}
		\end{framed}
		\item $\left(\frac{101}{97}\right)$
		\begin{framed}
		\vspace{1in}
		\end{framed}
		\item $\left(\frac{5}{103}\right)$
		\begin{framed}
		\vspace{1in}
		\end{framed}
		\item $\left(\frac{103}{5}\right)$
		\begin{framed}
		\vspace{1in}
		\end{framed}
		\item $\left(\frac{69}{389}\right)$
		\begin{framed}
		\vspace{1in}
		\end{framed}
		\end{enumerate}
\end{enumerate}


\section{Recommended Exercises}
\noindent These will not be graded but are recommended if you need more practice.
\begin{itemize}
	\item Section 3.13: \# 29, 30, 31, 32
	\item Section 3.14: \# 11, 12, 13
\end{itemize}
	
\end{document}
