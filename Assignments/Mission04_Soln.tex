\documentclass[12pt]{amsart}
\pagestyle{plain}

\usepackage{amsthm, setspace, framed, hyperref}
\usepackage[pdftex]{graphicx}
\usepackage{enumerate}

\usepackage[left=1in, right=1in, top=1in, bottom=1in]{geometry}
\setlength\parindent{0pt}

\theoremstyle{plain}
\newtheorem{thm}{Theorem}[section]
\newtheorem{lem}[thm]{Lemma}
\newtheorem*{cor}{Corollary}
\newtheorem{quest}{Question}
\theoremstyle{definition}
\newtheorem*{defn}{Definition}
\newtheorem*{ex}{Example}

\begin{document}
\title[]{Cryptography Mission 04 Solutions}
\begin{tabular*}{\textwidth}{@{\extracolsep{\fill}}l l}
MATH/CSCI 408  & Name: \rule{7cm}{0.5pt} \\
\hline\hline
\end{tabular*} \\
\maketitle

\begin{center}\textbf{Deadline: Thursday, 22 September 2016 at 3:05pm}\\

This mission covers Sections 3.1 and 3.3.
\end{center}

\section{Graded Problems}

\begin{enumerate}[1.]
	\item Let $F_1 = 1, F_2 = 1, F_{n+1} = F_n + F_{n-1}$ define the Fibonacci numbers $1,1,2,3,5, \cdots$.
		\begin{enumerate}[a.]
			\item List the first 15 Fibonacci numbers.
				\begin{framed}
				$F_1=1, F_2 = 1, F_3, = 2, F_4 = 3, F_5 = 5, F_6 = 8, F_7 = 13, F_8 = 21, F_9 = 34, F_{10} = 55, F_{11} = 89, F_{12} = 144, F_{13} = 233, F_{14} = 377, F_{15} = 610$ 
				\end{framed}
			\item Compute the greatest common divisor for the following pairs: $F_{10}$ and $F_7$, $F_6$ and $F_9$, $F_6$ and $F_{12}$, $F_{10}$ and $F_{13}$.
				\begin{framed}
				\begin{align*}
				\gcd(F_{10},F_{7}) &= \gcd(55,13) = 1\\
				\gcd(F_{6},F_{9}) &= \gcd(8,34) =  2\\
				\gcd(F_{6},F_{12}) &= \gcd(8,144) = 8\\
				\gcd(F_{10},F_{13}) &= \gcd(55,233) = 1\\
				\end{align*}
				\end{framed}
			\item Look at your previous examples. It turns out that $\gcd(F_m,F_n) = F_{\gcd(m,n)}$.  Write out \textbf{two} specific and detailed examples to verify that you believe this is true.
			\begin{framed}
			Note that $\gcd(F_{3},F_{4}) = \gcd(2,3) = 1$, $\gcd(3,4) = 1$, and $F_{1} = 1$.
			\end{framed}
			\begin{framed}
			Note that $\gcd(F_{3},F_{9}) = \gcd(2,34) = 2$, $\gcd(3,9) = 3$, and $F_{3} = 2$.
			\end{framed}
			\item Play with some examples, and make a conjecture about $\gcd(F_n,F_{n-1})$ for $n \geq 1$.  Are there any patterns?  Describe them here.
			\begin{framed}
			Some examples:
			\begin{align*}
				\gcd(F_{3},F_{4}) &= \gcd(2,3) = 1\\
				\gcd(F_{5},F_{6}) &= \gcd(5,8) =  1\\
				\gcd(F_{7},F_{8}) &= \gcd(13,21) = 1\\
				\end{align*}
				Conjecture: $\gcd(F_n,F_{n-1}) = 1$ since we always seem to be getting 1 so far.
			\end{framed}
		\end{enumerate}
	\item You can compute a gcd using SageMath's \texttt{gcd(a,b)}.  Determine the solution for the following gcd computations.
		\begin{enumerate}[a.]
			\item $\gcd(234,6013)$
			\begin{framed}
			1
			\end{framed}
			\item $\gcd(74951,26269)$
			\begin{framed}
			241
			\end{framed}
			\item $\gcd(5223389,188434513)$
			\begin{framed}
			30193
			\end{framed}
		\end{enumerate}
	\item In class, we started practicing writing proofs or formal mathematical arguments.  In this problem, we're going to walk through the proof of a theorem.
	\begin{enumerate}[a.]
		\item The Theorem you want to prove is: Let $a,b,c,d,$ and $n$ be integers with $n > 0$.  If $a \equiv b \bmod n$ and $c \equiv d \bmod n$, then $ac \equiv bd \mod n$.  First, come up with an example (with specific numbers) to convince yourself this is true.
		\begin{framed}
		Suppose we have the following numbers:
		\begin{align*}
		a &= 1\\
		b &= 6\\
		c &= 11\\
		d &= 16\\
		n &= 5\\
		\end{align*}
		Note that $1 \equiv 6 \bmod 5$ and $11 \equiv 16 \bmod 5$.  If we multiply, we get $ac = 1\cdot 11 = 11 \bmod 5 \equiv 1 \bmod 5$ as well as $bd = 6\cdot 16 = 96 \bmod 5 \equiv 1 \bmod 5$.
		\end{framed}
		\item Which part of the theorem is the hypothesis?  This is what you assume.
		\begin{framed}
		Let $a,b,c,d,$ and $n$ be integers with $n > 0$.  If $a \equiv b \bmod n$ and $c \equiv d \bmod n$...
		\end{framed}
		\item Which part of the theorem is the conclusion? This will be what you show is true based on the hypothesis.
		\begin{framed}
		... then $ac \equiv bd \mod n$.
		\end{framed}
		\item Write out the definition of $a \equiv b \bmod n$.
		\begin{framed}
		This means $n \mid (a-b)$ or $(a-b) = nk$ for some integer $k$.
		\end{framed}
		\item Now write the proof.  Start by assuming the hypothesis.  Use the necessary definitions and work your way towards the conclusion.
		\begin{framed}
		\begin{proof}Let $a,b,c,d,$ and $n$ be integers with $n > 0$.  If $a \equiv b \bmod n$ and $c \equiv d \bmod n$, then by definition, we have $n \mid (a-b)$ and $n \mid (c-d)$.  This implies $(a-b) = nk$ for some integer $k$, and $(c-d) = nl$ for some integer $l$.  Rewrite these two equations to get:
		\begin{align*}
		a &= nk+b\\
		c &= nl + d.
		\end{align*}
		Consider
		\begin{align*}
		ac &= (nk+b)(nl+d)\\
		&= n^2kl+nkd+bnl+bd\\
		ac - bd &= n(nkl+kd+bl.)
		\end{align*} 
		Since $nkl+kd+bl$ is just another integer, this means $n \mid (ac-bd)$ or that $$ac \equiv bd \bmod n.$$
		\end{proof}
		\end{framed}
	\end{enumerate}
\end{enumerate}

\end{document}
