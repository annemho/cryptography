\documentclass[12pt]{amsart}
\pagestyle{plain}

\usepackage{amsthm, setspace, framed, hyperref}
\usepackage[pdftex]{graphicx}
\usepackage{enumerate}
\usepackage[none]{hyphenat}
\usepackage{listings} %For code

\usepackage[left=1in, right=1in, top=1in, bottom=1in]{geometry}
\setlength\parindent{0pt}

\theoremstyle{plain}
\newtheorem{thm}{Theorem}[section]
\newtheorem{lem}[thm]{Lemma}
\newtheorem*{cor}{Corollary}
\newtheorem{quest}{Question}
\theoremstyle{definition}
\newtheorem*{defn}{Definition}
\newtheorem*{ex}{Example}

\begin{document}
\title[]{Cryptography Mission 09 Solutions}
\begin{tabular*}{\textwidth}{@{\extracolsep{\fill}}l l}
MATH/CSCI 408  & Name: \rule{7cm}{0.5pt} \\
\hline\hline
\end{tabular*} \\
\maketitle

\begin{center}\textbf{Deadline: Thursday, 17 November 2016 at 3:05pm}\\

This mission covers Sections 6.3, 6.4, and 7.1
\end{center}

\section{Graded Problems}

\begin{enumerate}[1.]
	\item Read through the Miller-Rabin Primality Test (6.3 p. 178).  Work through the example, and write a new example here.\\
		\begin{framed}
		\begin{align*}
		n &= 101\\
		n-1 &= 100 = 2^2\cdot 25\\
		\text{Choose } a &= 3.\\
		b_0 = 3^{25} \equiv 10 \bmod 101\\
		b_1 = 10^2 \equiv 100 \bmod 100\\
		\end{align*}
		So $n = 101$ is probably prime.\\
		\end{framed}
	\item Use the Fermat Factoring method to factor 70747.\\
		\begin{framed}
		\begin{align*}
		70747+1^2 &= 70748\\
		70747+2^2 &= 70751\\
		70747+3^2 &= 70756\\
		\end{align*}
		and $\sqrt{70756} = 266$, so $70756 = (266+3)(266-3) = 269\cdot 263$.
		\end{framed}
	\item Use the $p-1$ Factoring Algorithm to factor 4757.\\ 
		\begin{framed}
		\begin{lstlisting}
		n = 4757
		for i in range(1,10):
			print gcd(2^(factorial(i))%n-1,n)}
		\end{lstlisting}
		We find that when $i = 7$, we get the $\gcd$ to be 71, so 71 is a factor and so is $4757/71 = 67$.
		\end{framed}
	\item Use SageMath's \texttt{factor()} to check your answers to problems 1 and 2.\\
		\begin{framed}
		\begin{lstlisting}
		factor(101)
		factor(70747)
		\end{lstlisting}
		yields 101 (which is prime) and $263\cdot 269$.
		\end{framed}
	\item Given $p = 17$.  Solve the following discrete logs problems if possible.  If not, explain why.
		\begin{enumerate}[a.]
			\item $14 \equiv 3^x \mod 17$
			\begin{framed}
			$x = 9$ works.
			\end{framed}
			\newpage \item $5 \equiv 4^x \mod 17$
			\begin{framed}
			This is impossible.  If we try several $x$ values, we find that we get possible $\bmod 17$ values of 1, 4, 16, 13.  These repeat, so any other value (namely, 5) won't appear.
			\end{framed}
		\end{enumerate}
\end{enumerate}


\section{Recommended Exercises}
\noindent These will not be graded but are recommended if you need more practice.
\begin{itemize}
	\item Section 6.8: \# 9, 13, 18
	\item Section 6.9: \# 11
\end{itemize}
	
\end{document}
